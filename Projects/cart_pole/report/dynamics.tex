\documentclass[12pt]{article}
\usepackage{amsmath}
\usepackage{amssymb}
\usepackage[margin=1in]{geometry}

\title{Derivación Completa de las Ecuaciones de Euler-Lagrange\\
para el Péndulo Invertido sobre Carrito}
\author{Desarrollo Paso a Paso}
\date{}

\begin{document}

\maketitle

\section{Descripción del Sistema}

Consideramos un péndulo invertido montado sobre un carrito que se mueve horizontalmente:

\begin{itemize}
    \item \textbf{Carrito:} masa $M$, posición horizontal $p$
    \item \textbf{Péndulo:} masa $m$, longitud al centro de masa $\ell$, momento de inercia $I_{\text{pend}}$
    \item \textbf{Pivote:} ubicado a altura $h_{\text{pivot}}$ sobre el carrito
    \item \textbf{Control:} fuerza horizontal $F$ aplicada al carrito
    \item \textbf{Gravedad:} $g = 9.81 \, \text{m/s}^2$
\end{itemize}

\subsection{Coordenadas Generalizadas}

\begin{align}
q_1 &= p \quad \text{(posición horizontal del carrito)} \\
q_2 &= \theta \quad \text{(ángulo del péndulo con la vertical)} \\
\dot{q}_1 &= v = \dot{p} \quad \text{(velocidad del carrito)} \\
\dot{q}_2 &= \omega = \dot{\theta} \quad \text{(velocidad angular del péndulo)}
\end{align}

\section{Cinemática del Sistema}

\subsection{Posición del Centro de Masa del Péndulo}

El centro de masa del péndulo se encuentra a distancia $\ell$ del pivote. En coordenadas cartesianas:

\begin{align}
x_{\text{pend}} &= p + \ell \sin\theta \\
z_{\text{pend}} &= h_{\text{pivot}} - \ell \cos\theta
\end{align}

\textbf{Explicación:}
\begin{itemize}
    \item El pivote se mueve horizontalmente con el carrito en $p$
    \item El péndulo se desvía horizontalmente $\ell\sin\theta$ desde el pivote
    \item Verticalmente, el péndulo "cuelga" $\ell\cos\theta$ debajo del pivote
\end{itemize}

\subsection{Velocidad del Centro de Masa del Péndulo}

Derivamos las posiciones respecto al tiempo:

\subsubsection{Velocidad Horizontal}

\begin{align}
\dot{x}_{\text{pend}} &= \frac{d}{dt}[p + \ell\sin\theta] \\
&= \dot{p} + \ell \frac{d}{dt}[\sin\theta] \\
&= v + \ell \cos\theta \cdot \dot{\theta} \\
&= v + \ell \cos\theta \cdot \omega
\end{align}

\textbf{Nota:} Usamos la regla de la cadena: $\frac{d}{dt}[\sin\theta] = \cos\theta \cdot \dot{\theta}$

\subsubsection{Velocidad Vertical}

\begin{align}
\dot{z}_{\text{pend}} &= \frac{d}{dt}[h_{\text{pivot}} - \ell\cos\theta] \\
&= 0 - \ell \frac{d}{dt}[\cos\theta] \\
&= -\ell(-\sin\theta) \cdot \dot{\theta} \\
&= \ell \sin\theta \cdot \omega
\end{align}

\textbf{Nota:} $h_{\text{pivot}}$ es constante, y $\frac{d}{dt}[\cos\theta] = -\sin\theta \cdot \dot{\theta}$

\section{Energías del Sistema}

\subsection{Energía Cinética}

\subsubsection{Energía Cinética del Carrito}

El carrito solo tiene movimiento traslacional horizontal:

\begin{equation}
T_{\text{cart}} = \frac{1}{2}M v^2
\end{equation}

\subsubsection{Energía Cinética del Péndulo}

El péndulo tiene dos componentes:

\textbf{1. Traslación del centro de masa:}

\begin{align}
T_{\text{pend,trans}} &= \frac{1}{2}m(\dot{x}_{\text{pend}}^2 + \dot{z}_{\text{pend}}^2) \\
&= \frac{1}{2}m[(v + \ell\cos\theta \cdot \omega)^2 + (\ell\sin\theta \cdot \omega)^2]
\end{align}

Expandiendo el primer término:

\begin{align}
(v + \ell\cos\theta \cdot \omega)^2 &= v^2 + 2v\ell\cos\theta \cdot \omega + \ell^2\cos^2\theta \cdot \omega^2
\end{align}

Expandiendo el segundo término:

\begin{align}
(\ell\sin\theta \cdot \omega)^2 &= \ell^2\sin^2\theta \cdot \omega^2
\end{align}

Sumando ambos:

\begin{align}
\dot{x}_{\text{pend}}^2 + \dot{z}_{\text{pend}}^2 &= v^2 + 2v\ell\cos\theta \cdot \omega + \ell^2\cos^2\theta \cdot \omega^2 + \ell^2\sin^2\theta \cdot \omega^2 \\
&= v^2 + 2v\ell\cos\theta \cdot \omega + \ell^2(\cos^2\theta + \sin^2\theta)\omega^2 \\
&= v^2 + 2v\ell\cos\theta \cdot \omega + \ell^2\omega^2
\end{align}

\textbf{Identidad trigonométrica usada:} $\cos^2\theta + \sin^2\theta = 1$

Por lo tanto:

\begin{equation}
T_{\text{pend,trans}} = \frac{1}{2}m[v^2 + 2v\ell\cos\theta \cdot \omega + \ell^2\omega^2]
\end{equation}

\textbf{2. Rotación sobre su centro de masa:}

\begin{equation}
T_{\text{pend,rot}} = \frac{1}{2}I_{\text{pend}}\omega^2
\end{equation}

\textbf{Energía cinética total del péndulo:}

\begin{align}
T_{\text{pend}} &= T_{\text{pend,trans}} + T_{\text{pend,rot}} \\
&= \frac{1}{2}m[v^2 + 2v\ell\cos\theta \cdot \omega + \ell^2\omega^2] + \frac{1}{2}I_{\text{pend}}\omega^2 \\
&= \frac{1}{2}mv^2 + mv\ell\cos\theta \cdot \omega + \frac{1}{2}(m\ell^2 + I_{\text{pend}})\omega^2
\end{align}

\subsubsection{Energía Cinética Total del Sistema}

\begin{align}
T &= T_{\text{cart}} + T_{\text{pend}} \\
&= \frac{1}{2}Mv^2 + \frac{1}{2}mv^2 + mv\ell\cos\theta \cdot \omega + \frac{1}{2}(m\ell^2 + I_{\text{pend}})\omega^2 \\
&= \frac{1}{2}(M + m)v^2 + mv\ell\cos\theta \cdot \omega + \frac{1}{2}J\omega^2
\end{align}

donde definimos el \textbf{momento de inercia efectivo} del péndulo sobre el pivote:

\begin{equation}
J = m\ell^2 + I_{\text{pend}}
\end{equation}

\textbf{Nota:} Este es el teorema de los ejes paralelos (Steiner).

\subsection{Energía Potencial}

\subsubsection{Energía Potencial del Carrito}

Si el centro de masa del carrito está a altura constante $h_c$:

\begin{equation}
U_{\text{cart}} = Mgh_c
\end{equation}

Esta es una \textbf{constante}, por lo que no afectará las ecuaciones de movimiento.

\subsubsection{Energía Potencial del Péndulo}

La altura del centro de masa del péndulo es:

\begin{equation}
U_{\text{pend}} = mg \cdot z_{\text{pend}} = mg(h_{\text{pivot}} - \ell\cos\theta)
\end{equation}

El término $mgh_{\text{pivot}}$ es constante, así que podemos escribir:

\begin{equation}
U_{\text{pend}} = -mg\ell\cos\theta + \text{constante}
\end{equation}

\subsubsection{Energía Potencial Total}

Ignorando las constantes:

\begin{equation}
U = -mg\ell\cos\theta
\end{equation}

\section{Lagrangiano del Sistema}

El Lagrangiano se define como:

\begin{equation}
\mathcal{L} = T - U
\end{equation}

Sustituyendo:

\begin{align}
\mathcal{L} &= \frac{1}{2}(M + m)v^2 + mv\ell\cos\theta \cdot \omega + \frac{1}{2}J\omega^2 - (-mg\ell\cos\theta) \\
&= \frac{1}{2}(M + m)v^2 + mv\ell\cos\theta \cdot \omega + \frac{1}{2}J\omega^2 + mg\ell\cos\theta
\end{align}

\textbf{Forma final del Lagrangiano:}

\begin{equation}
\boxed{\mathcal{L} = \frac{1}{2}(M + m)v^2 + mv\ell\cos\theta \cdot \omega + \frac{1}{2}J\omega^2 + mg\ell\cos\theta}
\end{equation}

\section{Ecuaciones de Euler-Lagrange}

Las ecuaciones de Euler-Lagrange son:

\begin{equation}
\frac{d}{dt}\left(\frac{\partial \mathcal{L}}{\partial \dot{q}_i}\right) - \frac{\partial \mathcal{L}}{\partial q_i} = Q_i
\end{equation}

donde $Q_i$ son las fuerzas generalizadas.

\subsection{Ecuación para la Coordenada $p$ (Posición del Carrito)}

\subsubsection{Paso 1: Calcular $\frac{\partial \mathcal{L}}{\partial v}$}

El Lagrangiano es:

\begin{equation}
\mathcal{L} = \frac{1}{2}(M + m)v^2 + mv\ell\cos\theta \cdot \omega + \frac{1}{2}J\omega^2 + mg\ell\cos\theta
\end{equation}

Los términos que contienen $v$ son:

\begin{equation}
\frac{1}{2}(M + m)v^2 \quad \text{y} \quad mv\ell\cos\theta \cdot \omega
\end{equation}

Derivando parcialmente con respecto a $v$:

\begin{align}
\frac{\partial \mathcal{L}}{\partial v} &= \frac{\partial}{\partial v}\left[\frac{1}{2}(M + m)v^2\right] + \frac{\partial}{\partial v}[mv\ell\cos\theta \cdot \omega] \\
&= (M + m)v + m\ell\cos\theta \cdot \omega
\end{align}

\textbf{Resultado:}

\begin{equation}
\boxed{\frac{\partial \mathcal{L}}{\partial v} = (M + m)v + m\ell\cos\theta \cdot \omega}
\end{equation}

\subsubsection{Paso 2: Calcular $\frac{d}{dt}\left(\frac{\partial \mathcal{L}}{\partial v}\right)$}

Derivamos con respecto al tiempo:

\begin{align}
\frac{d}{dt}\left(\frac{\partial \mathcal{L}}{\partial v}\right) &= \frac{d}{dt}[(M + m)v + m\ell\cos\theta \cdot \omega]
\end{align}

Aplicando la derivada término a término:

\textbf{Primer término:}

\begin{equation}
\frac{d}{dt}[(M + m)v] = (M + m)\dot{v} = (M + m)a
\end{equation}

donde $a = \ddot{p}$ es la aceleración del carrito.

\textbf{Segundo término:}

\begin{equation}
\frac{d}{dt}[m\ell\cos\theta \cdot \omega] = m\ell \frac{d}{dt}[\cos\theta \cdot \omega]
\end{equation}

Usando la regla del producto: $\frac{d}{dt}[f \cdot g] = \dot{f} \cdot g + f \cdot \dot{g}$

\begin{align}
\frac{d}{dt}[\cos\theta \cdot \omega] &= \frac{d}{dt}[\cos\theta] \cdot \omega + \cos\theta \cdot \frac{d}{dt}[\omega] \\
&= (-\sin\theta \cdot \dot{\theta}) \cdot \omega + \cos\theta \cdot \dot{\omega} \\
&= -\sin\theta \cdot \omega^2 + \cos\theta \cdot \alpha
\end{align}

donde $\alpha = \ddot{\theta}$ es la aceleración angular.

Por lo tanto:

\begin{equation}
\frac{d}{dt}[m\ell\cos\theta \cdot \omega] = m\ell(-\sin\theta \cdot \omega^2 + \cos\theta \cdot \alpha)
\end{equation}

\textbf{Sumando ambos términos:}

\begin{align}
\frac{d}{dt}\left(\frac{\partial \mathcal{L}}{\partial v}\right) &= (M + m)a + m\ell\cos\theta \cdot \alpha - m\ell\sin\theta \cdot \omega^2
\end{align}

\textbf{Resultado:}

\begin{equation}
\boxed{\frac{d}{dt}\left(\frac{\partial \mathcal{L}}{\partial v}\right) = (M + m)a + m\ell\cos\theta \cdot \alpha - m\ell\sin\theta \cdot \omega^2}
\end{equation}

\subsubsection{Paso 3: Calcular $\frac{\partial \mathcal{L}}{\partial p}$}

Revisamos el Lagrangiano:

\begin{equation}
\mathcal{L} = \frac{1}{2}(M + m)v^2 + mv\ell\cos\theta \cdot \omega + \frac{1}{2}J\omega^2 + mg\ell\cos\theta
\end{equation}

\textbf{Ningún término depende explícitamente de $p$}, por lo tanto:

\begin{equation}
\boxed{\frac{\partial \mathcal{L}}{\partial p} = 0}
\end{equation}

\subsubsection{Paso 4: Fuerza Generalizada $Q_p$}

La fuerza externa $F$ actúa horizontalmente sobre el carrito, por lo que:

\begin{equation}
Q_p = F
\end{equation}

\subsubsection{Paso 5: Aplicar Euler-Lagrange}

\begin{equation}
\frac{d}{dt}\left(\frac{\partial \mathcal{L}}{\partial v}\right) - \frac{\partial \mathcal{L}}{\partial p} = Q_p
\end{equation}

Sustituyendo:

\begin{equation}
(M + m)a + m\ell\cos\theta \cdot \alpha - m\ell\sin\theta \cdot \omega^2 - 0 = F
\end{equation}

\textbf{Primera ecuación del sistema:}

\begin{equation}
\boxed{(M + m)\ddot{p} + m\ell\cos\theta \cdot \ddot{\theta} - m\ell\sin\theta \cdot \dot{\theta}^2 = F}
\end{equation}

O reordenando:

\begin{equation}
\boxed{(M + m)\ddot{p} + m\ell\cos\theta \cdot \ddot{\theta} = F + m\ell\sin\theta \cdot \dot{\theta}^2}
\end{equation}

\subsection{Ecuación para la Coordenada $\theta$ (Ángulo del Péndulo)}

\subsubsection{Paso 1: Calcular $\frac{\partial \mathcal{L}}{\partial \omega}$}

Los términos que contienen $\omega = \dot{\theta}$ son:

\begin{equation}
mv\ell\cos\theta \cdot \omega \quad \text{y} \quad \frac{1}{2}J\omega^2
\end{equation}

Derivando parcialmente:

\begin{align}
\frac{\partial \mathcal{L}}{\partial \omega} &= \frac{\partial}{\partial \omega}[mv\ell\cos\theta \cdot \omega] + \frac{\partial}{\partial \omega}\left[\frac{1}{2}J\omega^2\right] \\
&= mv\ell\cos\theta + J\omega
\end{align}

\textbf{Resultado:}

\begin{equation}
\boxed{\frac{\partial \mathcal{L}}{\partial \omega} = mv\ell\cos\theta + J\omega}
\end{equation}

\subsubsection{Paso 2: Calcular $\frac{d}{dt}\left(\frac{\partial \mathcal{L}}{\partial \omega}\right)$}

\begin{equation}
\frac{d}{dt}\left(\frac{\partial \mathcal{L}}{\partial \omega}\right) = \frac{d}{dt}[mv\ell\cos\theta + J\omega]
\end{equation}

\textbf{Primer término:}

\begin{equation}
\frac{d}{dt}[mv\ell\cos\theta] = m\ell \frac{d}{dt}[v\cos\theta]
\end{equation}

Usando la regla del producto:

\begin{align}
\frac{d}{dt}[v\cos\theta] &= \dot{v}\cos\theta + v \frac{d}{dt}[\cos\theta] \\
&= a\cos\theta + v(-\sin\theta \cdot \omega) \\
&= a\cos\theta - v\sin\theta \cdot \omega
\end{align}

Por lo tanto:

\begin{equation}
\frac{d}{dt}[mv\ell\cos\theta] = m\ell(a\cos\theta - v\sin\theta \cdot \omega)
\end{equation}

\textbf{Segundo término:}

\begin{equation}
\frac{d}{dt}[J\omega] = J\dot{\omega} = J\alpha
\end{equation}

\textbf{Sumando:}

\begin{align}
\frac{d}{dt}\left(\frac{\partial \mathcal{L}}{\partial \omega}\right) &= m\ell a\cos\theta - m\ell v\sin\theta \cdot \omega + J\alpha
\end{align}

\textbf{Resultado:}

\begin{equation}
\boxed{\frac{d}{dt}\left(\frac{\partial \mathcal{L}}{\partial \omega}\right) = m\ell\cos\theta \cdot a - m\ell\sin\theta \cdot v\omega + J\alpha}
\end{equation}

\subsubsection{Paso 3: Calcular $\frac{\partial \mathcal{L}}{\partial \theta}$}

Los términos que contienen $\theta$ (no $\omega$) son:

\begin{equation}
mv\ell\cos\theta \cdot \omega \quad \text{y} \quad mg\ell\cos\theta
\end{equation}

Derivando parcialmente:

\begin{align}
\frac{\partial \mathcal{L}}{\partial \theta} &= \frac{\partial}{\partial \theta}[mv\ell\cos\theta \cdot \omega] + \frac{\partial}{\partial \theta}[mg\ell\cos\theta] \\
&= mv\ell(-\sin\theta) \cdot \omega + mg\ell(-\sin\theta) \\
&= -m\ell\sin\theta(v\omega + g)
\end{align}

\textbf{Resultado:}

\begin{equation}
\boxed{\frac{\partial \mathcal{L}}{\partial \theta} = -m\ell\sin\theta \cdot v\omega - mg\ell\sin\theta}
\end{equation}

\subsubsection{Paso 4: Fuerza Generalizada $Q_\theta$}

No hay par externo aplicado directamente al péndulo:

\begin{equation}
Q_\theta = 0
\end{equation}

\subsubsection{Paso 5: Aplicar Euler-Lagrange}

\begin{equation}
\frac{d}{dt}\left(\frac{\partial \mathcal{L}}{\partial \omega}\right) - \frac{\partial \mathcal{L}}{\partial \theta} = Q_\theta
\end{equation}

Sustituyendo:

\begin{align}
&m\ell\cos\theta \cdot a - m\ell\sin\theta \cdot v\omega + J\alpha \\
&\quad - (-m\ell\sin\theta \cdot v\omega - mg\ell\sin\theta) = 0
\end{align}

Simplificando:

\begin{align}
&m\ell\cos\theta \cdot a - m\ell\sin\theta \cdot v\omega + J\alpha \\
&\quad + m\ell\sin\theta \cdot v\omega + mg\ell\sin\theta = 0
\end{align}

\textbf{Los términos $m\ell\sin\theta \cdot v\omega$ se cancelan:}

\begin{equation}
m\ell\cos\theta \cdot a + J\alpha + mg\ell\sin\theta = 0
\end{equation}

\textbf{Segunda ecuación del sistema:}

\begin{equation}
\boxed{m\ell\cos\theta \cdot \ddot{p} + J\ddot{\theta} + mg\ell\sin\theta = 0}
\end{equation}

O reordenando:

\begin{equation}
\boxed{J\ddot{\theta} + m\ell\cos\theta \cdot \ddot{p} = -mg\ell\sin\theta}
\end{equation}

\section{Sistema de Ecuaciones Acopladas}

Tenemos dos ecuaciones con dos incógnitas ($\ddot{p}$ y $\ddot{\theta}$):

\begin{align}
(M + m)\ddot{p} + m\ell\cos\theta \cdot \ddot{\theta} &= F + m\ell\sin\theta \cdot \dot{\theta}^2 \tag{Ec. 1}\\
m\ell\cos\theta \cdot \ddot{p} + J\ddot{\theta} &= -mg\ell\sin\theta \tag{Ec. 2}
\end{align}

\subsection{Forma Matricial}

\begin{equation}
\begin{bmatrix}
M + m & m\ell\cos\theta \\
m\ell\cos\theta & J
\end{bmatrix}
\begin{bmatrix}
\ddot{p} \\
\ddot{\theta}
\end{bmatrix}
=
\begin{bmatrix}
F + m\ell\sin\theta \cdot \dot{\theta}^2 \\
-mg\ell\sin\theta
\end{bmatrix}
\end{equation}

O de forma compacta:

\begin{equation}
\mathbf{M}(\theta) \ddot{\mathbf{q}} = \mathbf{b}(\theta, \dot{\theta}, F)
\end{equation}

\section{Solución del Sistema (Regla de Cramer)}

\subsection{Determinante de la Matriz de Masa}

\begin{align}
\text{det}(\mathbf{M}) &= (M + m) \cdot J - (m\ell\cos\theta)^2 \\
&= (M + m)J - m^2\ell^2\cos^2\theta
\end{align}

\textbf{Definimos:}

\begin{equation}
\boxed{\Delta = (M + m)J - m^2\ell^2\cos^2\theta}
\end{equation}

\subsection{Solución para $\ddot{p}$}

Usando la regla de Cramer:

\begin{equation}
\ddot{p} = \frac{1}{\Delta}
\begin{vmatrix}
F + m\ell\sin\theta \cdot \dot{\theta}^2 & m\ell\cos\theta \\
-mg\ell\sin\theta & J
\end{vmatrix}
\end{equation}

Calculando el determinante:

\begin{align}
&= \frac{1}{\Delta}\left[J(F + m\ell\sin\theta \cdot \dot{\theta}^2) - m\ell\cos\theta \cdot (-mg\ell\sin\theta)\right] \\
&= \frac{1}{\Delta}\left[JF + Jm\ell\sin\theta \cdot \dot{\theta}^2 + m^2g\ell^2\sin\theta\cos\theta\right] \\
&= \frac{1}{\Delta}\left[JF + m\ell\sin\theta(J\dot{\theta}^2 + mg\ell\cos\theta)\right]
\end{align}

\textbf{Resultado:}

\begin{equation}
\boxed{\ddot{p} = \frac{JF + m\ell\sin\theta(J\dot{\theta}^2 + mg\ell\cos\theta)}{(M+m)J - m^2\ell^2\cos^2\theta}}
\end{equation}

\subsection{Solución para $\ddot{\theta}$}

Usando la regla de Cramer:

\begin{equation}
\ddot{\theta} = \frac{1}{\Delta}
\begin{vmatrix}
M + m & F + m\ell\sin\theta \cdot \dot{\theta}^2 \\
m\ell\cos\theta & -mg\ell\sin\theta
\end{vmatrix}
\end{equation}

Calculando el determinante:

\begin{align}
&= \frac{1}{\Delta}\left[(M+m)(-mg\ell\sin\theta) - m\ell\cos\theta(F + m\ell\sin\theta \cdot \dot{\theta}^2)\right] \\
&= \frac{1}{\Delta}\left[-(M+m)mg\ell\sin\theta - m\ell\cos\theta \cdot F - m^2\ell^2\sin\theta\cos\theta \cdot \dot{\theta}^2\right]
\end{align}

\textbf{Resultado:}

\begin{equation}
\boxed{\ddot{\theta} = \frac{-m\ell\cos\theta \cdot F - (M+m)mg\ell\sin\theta - m^2\ell^2\sin\theta\cos\theta \cdot \dot{\theta}^2}{(M+m)J - m^2\ell^2\cos^2\theta}}
\end{equation}

\section{Forma Explícita de las Ecuaciones Dinámicas}

\subsection{Notación Simplificada}

Definimos:
\begin{align}
s &= \sin\theta \\
c &= \cos\theta \\
\Delta &= (M+m)J - m^2\ell^2c^2
\end{align}

\subsection{Ecuaciones Finales}

\begin{equation}
\boxed{
\begin{aligned}
\ddot{p} &= \frac{JF + m\ell s(J\dot{\theta}^2 + mg\ell c)}{\Delta} \\[10pt]
\ddot{\theta} &= \frac{-m\ell c \cdot F - (M+m)mg\ell s - m^2\ell^2 sc \cdot \dot{\theta}^2}{\Delta}
\end{aligned}
}
\end{equation}

Estas son las ecuaciones que implementas en tu modelo de Acados.

\section{Sistema de Estados}

Para implementación numérica, definimos el vector de estados:

\begin{equation}
\mathbf{x} = 
\begin{bmatrix}
p \\
\theta \\
v \\
\omega
\end{bmatrix}
=
\begin{bmatrix}
p \\
\theta \\
\dot{p} \\
\dot{\theta}
\end{bmatrix}
\end{equation}

Y el sistema de primer orden:

\begin{equation}
\dot{\mathbf{x}} = 
\begin{bmatrix}
\dot{p} \\
\dot{\theta} \\
\ddot{p} \\
\ddot{\theta}
\end{bmatrix}
=
\begin{bmatrix}
v \\
\omega \\
\frac{JF + m\ell\sin\theta(J\omega^2 + mg\ell\cos\theta)}{\Delta} \\[5pt]
\frac{-m\ell\cos\theta \cdot F - (M+m)mg\ell\sin\theta - m^2\ell^2\sin\theta\cos\theta \cdot \omega^2}{\Delta}
\end{bmatrix}
\end{equation}

\section{Resumen de Parámetros}

\begin{itemize}
    \item $M$: masa del carrito [kg]
    \item $m$: masa del péndulo [kg]
    \item $\ell$: distancia del pivote al centro de masa del péndulo [m]
    \item $I_{\text{pend}}$: momento de inercia del péndulo sobre su centro de masa [kg$\cdot$m$^2$]
    \item $J = m\ell^2 + I_{\text{pend}}$: momento de inercia efectivo sobre el pivote [kg$\cdot$m$^2$]
    \item $g = 9.81$: aceleración gravitacional [m/s$^2$]
    \item $F$: fuerza de control aplicada al carrito [N]
\end{itemize}

\end{document}