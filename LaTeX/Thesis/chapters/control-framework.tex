\section{Control and Optimization framework}
acados is an open-source software framework for solving optimal control problems with a strong focus on computational efficiency and real-time applicability. It is primarily designed for nonlinear model predictive control (NMPC) and moving horizon estimation (MHE), making it well suited for robotics and control applications where fast and reliable optimization is required.

At its core, acados provides highly optimized numerical solvers based on state-of-the-art algorithms such as Sequential Quadratic Programming (SQP) and interior-point methods, combined with tailored quadratic programming (QP) solvers. The framework is implemented in C in order to ensure high computational performance, efficient memory management, and suitability for real-time and embedded systems.

The formulation of the optimal control problem in acados is carried out using CasADi, which serves as a symbolic modeling and automatic differentiation engine. CasADi is employed to define the system dynamics, cost functions, and constraints in a compact and high-level manner. From these symbolic expressions, exact first- and second-order derivatives are automatically generated, which are essential for the efficiency and numerical robustness of gradient-based optimal control algorithms.

acados provides a dedicated interface to MATLAB that allows users to define and configure optimal control problems directly within the MATLAB environment. Through this interface, the problem description is translated into CasADi symbolic graphs, after which acados automatically generates, compiles, and links efficient solver code. This workflow enables rapid prototyping and debugging using MATLAB’s visualization and analysis tools, while retaining the computational performance of the underlying C-based solvers.

The choice of acados in this thesis is motivated by several key factors. First, its ability to solve nonlinear optimal control problems at high update rates makes it suitable for control tasks involving fast system dynamics and strict real-time constraints. Second, the use of automatic differentiation through CasADi significantly reduces development effort and minimizes the risk of analytical or implementation errors in derivative computations. Third, the MATLAB interface facilitates fast testing, tuning, and validation of control strategies in simulation prior to deployment. Finally, acados has been extensively validated in both academic and industrial contexts, demonstrating its reliability and suitability for advanced control and robotics applications.
