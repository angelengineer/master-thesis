\section{state-of-the-art}
This section reviews existing literature and prior work relevant to the development of the control framework for autobalancing robot platforms with a focus on two-wheeled inverted pendulum systems.

The two-wheeled inverted pendulum robot has been extensively studied due to its inherent instability and the challenges it presents for control system design. Several control strategies and structural variations has been studied.


- In \cite{onkolAdaptiveModelPredictive2018} the authors ddress the control of a two-wheeled balancing robot (TWBR) integrated with a four-link active manipulator, a configuration characterized by high nonlinearity and underactuation23. To manage the modeling complexity of the multi-link attachment, the authors employ a virtual link assumption, reducing the structure to a 2-DOF inverted pendulum model where the mass and center of gravity (COG) shift according to the manipulator's configuration and payload3.... The proposed control architecture utilizes a dual-loop framework that decouples the fast angular dynamics (inner loop) from the slower linear position dynamics (outer loop)67. While the outer loop is regulated by a standard Linear Quadratic Gaussian (LQG) controller, the inner loop employs an Adaptive Model Predictive Control (AMPC) strategy based on a Linear Parameter Varying (LPV) formulation2.... This LPV approach utilizes local linear models scheduled by the varying mass parameter to update the predictive model in real-time, complemented by a Time-Varying Kalman Filter (TVKF) to maintain estimator consistency as the plant dynamics shift. The significance of this work lies in its robust handling of discrete mass variations—representative of pick-and-place operations—which typically degrade the performance of static-gain controllers2.... By explicitly updating the prediction model and penalizing constraint violations within a quadratic programming framework, the AMPC scheme demonstrates superior reference tracking and disturbance rejection compared to proportional-integral-derivative (PID) and feedback linearization (FBL) benchmarks2.... However, the reliance on a virtual link abstraction assumes that the internal manipulator dynamics are sufficiently fast or compensated for, which may present a trade-off in high-fidelity human–robot interaction scenarios where unmodeled link-level reaction forces could perturb the balancing base3.... Nonetheless, the integration of LPV-based adaptation within a predictive control law provides a rigorous template for managing the parametric uncertainties inherent in mobile platforms with dynamic payloads8...

- In \cite{nghiaAdaptiveNeuralSliding2022} Nghia et al. address the trajectory tracking problem for a two-wheeled self-balancing robot (TWSBR) by proposing an adaptive sliding mode control (ASMC) architecture integrated with a radial basis function (RBF) neural network1. The platform is characterized by a three-degree-of-freedom reduced dynamic model derived from a nonholonomic Lagrange formulation, capturing the inherent coupling between yaw, position, and the underactuated tilt angle2.... Central to this approach is the online estimation of unknown dynamic parameters—encompassing centrifugal, Coriolis, and gravitational forces—through the RBF network, which effectively relaxes the requirement for a precise nominal model of the nonlinear plant15. Stability is established using the Lyapunov stability theorem and Barbalat’s lemma, ensuring that tracking errors converge to an asymptotic neighborhood of zero despite the presence of bounded external disturbances and modeling uncertainties5....
While the inclusion of a robust compensator to mitigate neural network estimation errors provides a robust framework for handling general dynamic uncertainties, the model assumes a relatively static mechanical configuration and does not explicitly account for variable payloads or shifts in the center of mass89. Furthermore, the implementation utilizes a hyperbolic tangent function to mitigate the chattering phenomenon inherent in sliding mode control, reflecting a design trade-off between control smoothness and high-frequency robust performance10. In the context of human–robot interaction, the proposed strategy focuses primarily on trajectory precision and balance stability1112; however, it lacks the explicit safety-critical constraints or admittance control layers necessary for managing physical interaction with humans. Consequently, the work serves as a significant reference for online parameter identification in underactuated systems but highlights a gap in addressing the specific time-varying mass properties and interaction-dependent forces relevant to collaborative robotic platforms.

- in \cite{nguyenAdaptiveNonlinearPD2024} propose an Adaptive Nonlinear Proportional-Derivative (ANPD) control scheme for a two-wheeled self-balancing (TWSB) robot, modeled as a high-order, underactuated mechanical system12. The mechanical structure is represented via the Lagrange equation, assuming a cylindrical body of mass M 
c
​
  and radius R 
c
​
  attached to the central axis of two identical wheels3.... The system’s state is defined by three generalized coordinates—bilateral wheel angles and the pitch angle—driven by only two independent motor torques, which introduces the inherent underactuation challenge45. To manage state estimation, the platform utilizes an MPU6050 IMU and wheel encoders, with the control logic executed on an STM32f4 platform at a 1~ms sampling rate67. This modeling framework explicitly assumes the robot's centroid is located on the central axis at a fixed height h, and the stability analysis leverages the skew-symmetric property of the ( 
J
˙
e
​
 −2F 
e
​
 ) matrix to ensure theoretical robustness3....
The proposed control architecture utilizes a nested loop structure where an outer position loop generates setpoints for an inner balancing loop1011. The ANPD formulation distinguishes itself by utilizing time-varying, nonlinear gains (K 
p
​
 (e) and K 
d
​
 ( e
˙ )) that adapt based on the magnitude of the tracking error relative to predefined thresholds, theoretically ensuring asymptotic stability via a Lyapunov candidate function that incorporates a radially unbounded integral term12.... A Genetic Algorithm (GA) is employed offline to optimize initial parameters, effectively addressing the traditional design trade-off between transient response and the manual tuning effort required for nonlinear systems1.... This approach is highly relevant to platforms operating under dynamic uncertainties and variable payloads, as experimental validation demonstrated superior rejection of external impulsive forces (modeled via a 0.4~kg mass impact) compared to standard PD and fixed-gain nonlinear PD controllers17.... While the method offers a computationally lightweight alternative to learning-based or fuzzy-logic strategies, its current formulation assumes nominal kinematic parameters, suggesting a potential limitation when facing significant parameter drift or large-scale payload variations typical in complex human-robot interaction scenarios

- \cite{mishraControlTwoWheelSelfBalancing2024} Mishra and Bansal12 investigate the stabilization and trajectory tracking of a two-wheeled self-balancing robot, an underactuated and inherently non-linear system, by comparing Linear Quadratic Regulator (LQR) and Model Predictive Control (MPC) strategies. The mechanical platform is modeled as a rigid chassis acting as an inverted pendulum coupled with dual DC motor actuators34. The control-oriented model is a linearized state-space representation derived using small-angle approximations (cosθ≈1,sinθ≈θ) around the vertical upward equilibrium4.... While the LQR framework minimizes a quadratic cost function over an infinite time horizon to derive an optimal state-feedback law78, the MPC implementation repeatedly solves an optimization problem over a finite prediction horizon (j=12) and control horizon (N=5) to handle system constraints and future behavior9....
Simulation results demonstrate that the MPC approach achieves superior performance, significantly reducing settling time and peak overshoot for both displacement and tilt states while requiring lower peak control effort compared to LQR1213. Despite these advantages, the modeling framework assumes static inertial parameters and does not explicitly account for variable payloads or the dynamic uncertainties inherent in interactive environments614. Robustness is evaluated through external impulse and sinusoidal disturbances rather than parametric variations or center-of-mass shifts1415. This highlights a critical contrast with platforms requiring human–robot interaction; whereas this study emphasizes optimized performance for a nominal system, it lacks the adaptive or robust formulations necessary to maintain stability under the significant structural uncertainties and varying load conditions characteristic of collaborative robotic platforms.

- \cite{zhangControlStrategiesTwoWheeled2025} Zhang and Mohamad Nor provide a comprehensive taxonomy of control strategies for two-wheeled self-balancing robots (TWSBRs), which are fundamentally characterized as underactuated, non-minimum phase mechanical systems with coupled longitudinal and tilt dynamics12. The authors evaluate physics-based modeling approaches—including Newtonian, Lagrangian, and Kane’s methods—emphasizing that while linearization around the vertical equilibrium facilitates the use of classical linear time-invariant controllers like PID and LQR, such approximations are locally constrained and often fail to address the nonlinearities inherent in large-angle excursions or high-speed maneuvers3.... To mitigate the dynamic uncertainties and variable payloads common in human-centric environments, the review identifies adaptive observers and high-gain estimation techniques as critical for real-time identification of undisclosed body weights and center-of-mass (CoM) shifts89.
From a control-theoretic perspective, the survey highlights a significant design trade-off between the formal stability of robust nonlinear methods, such as Sliding Mode Control (SMC), and the adaptability of intelligent paradigms like Type-II Fuzzy Logic and Reinforcement Learning7.... While SMC variants ensure finite-time convergence and insensitivity to matched uncertainties, they frequently introduce high-frequency chattering that can excite unmodeled dynamics; conversely, intelligent controllers offer superior noise immunity and handle unstructured environments more effectively but often demand prohibitive computational resources or lack rigorous stability guarantees11.... For platforms intended for human–robot interaction, Zhang and Mohamad Nor advocate for hybrid architectures—such as neural-network-enhanced adaptive controllers or LMI-based robust schemes—which combine the perception-driven adaptability of AI with the deterministic safety bounds of modern control theory to navigate the kinematic constraints of coaxially wheeled structures

\cite{isdaryaniDesignImplementationTwoWheeled} Isdaryani et al. explore the stabilization of an underactuated two-wheeled inverted pendulum (WIP) using Model Reference Adaptive Control (MRAC) to address inherent non-linearities and environmental disturbances1.... The platform is modeled via Newton-Euler equations, though the authors adopt a simplified second-order state-space representation by neglecting cart translation (x, 
x
˙
 ) to focus exclusively on the tilt dynamics (ϕ, 
ϕ
˙
​
 )45. Actuation is provided by two DC motors via an H-bridge driver, while sensing relies on an MPU6050 accelerometer-gyroscope fused through a Kalman filter to mitigate measurement noise6.... This design choice prioritizes balancing stability over trajectory tracking, operating under the assumption that rotational equilibrium is the primary control objective when the system is subjected to varying loads59.
The control architecture utilizes a Lyapunov-based MRAC scheme where three adjustable gain parameters (θ 
1
​
 ,θ 
2
​
 ,θ 
3
​
 ) are updated via an adaptation mechanism to minimize the tracking error between the physical plant and a second-order linear reference model10.... Unlike static PID or LQG methods, this adaptive approach ensures asymptotic stability and convergence of adaptation errors through the derivation of a symmetric positive definite matrix P via the Lyapunov equation3.... While the system demonstrates effective recovery from external manual disturbances, the reliance on a simplified 2nd-order reference model and the omission of translational states may present trade-offs in human--robot interaction scenarios where cart-pendulum coupling and precise positioning are critical17.... In this framework, the reference model acts as a virtual scaffolding, with the controller continuously reshaping the system's gains to ensure the physical hardware "haunts" the ideal trajectory of the model.

 - \cite{almeshalDynamicModellingStabilization2013} Almeshal et al.1 present a novel five-degree-of-freedom (DOF) self-balancing platform modeled as a double inverted pendulum on a differential-drive base. A key design departure from standard wheeled inverted pendulums is the inclusion of a linear actuator in the distal link, which introduces a translational DOF for active payload positioning and height adjustment12. The system's high nonlinearity is captured through a Lagrangian formulation that explicitly incorporates Coulomb joint friction and time-varying center of mass (COM) dynamics, which are critical for representing moving payloads1.... To stabilize this multi-input multi-output (MIMO) system, a hybrid control strategy is employed, utilizing PD controllers for wheel and proximal link regulation alongside PID controllers for the distal link and payload displacement to mitigate steady-state errors15.
The framework is particularly relevant to systems operating under variable payloads and dynamic uncertainties, as it provides an extensive analysis of robustness against external disturbances up to 300~N and varied perturbation durations6.... By modeling the payload as a point mass at a variable distance, the authors demonstrate how translational movement shifts the overall system equilibrium, a factor often neglected in simpler self-balancing models9.... However, the control strategy is fundamentally based on a decoupled feedback approach with heuristically tuned gains, which may limit its efficacy in handling the complex, non-conservative interaction forces typical of active human-robot collaboration5.... This architecture serves as a robust baseline for evaluating the trade-offs between mechanical flexibility and the control effort required to stabilize higher-DOF underactuated platforms.

- \cite{klokowskiEvoBOTDesignLearningBased2023} Klokowski et al.1 introduce the evoBOT, an underactuated mobile platform utilizing an inverse compound pendulum architecture characterized by six revolute joints and direct-drive brushless actuation to eliminate gearbox backlash2.... Unlike conventional wheeled inverted pendulum (WIP) systems, the platform features a dual-mode topology: an actively balanced pendulum mode and a statically stable support mode facilitated by arms with omnidirectional grippers25. Payload handling is modeled through a virtual pendulum abstraction, where the control system must compensate for tilting moments induced by normal forces from the body, arms, and variable loads up to 30~kg35. Sensing is integrated via a 6-DOF inertial measurement unit and absolute encoders, supplemented by 360-degree cameras for localization, which obviates the need for mechanical gimbal stabilization24.
The control strategy deviates from traditional linear or robust formulations, employing a learning-based approach centered on Proximal Policy Optimization (PPO)16. The policy maps a high-dimensional state space—comprising base velocities, pitch orientation, and joint kinematics—to continuous wheel velocity targets tracked by low-level PD controllers78. To address dynamic uncertainties and ensure robust sim-to-real transfer, the authors implement an extensive pipeline of domain randomization, varying link masses by ±20\%, motor damping by ±50\%, and injecting Gaussian noise and delays up to 10ms9.... While the RL framework demonstrates a superior capacity for highly dynamic locomotion and recovery from external disturbances compared to quasi-static models, the approach relies on high-frequency inference (200Hz) and lacks formal Lyapunov-based stability guarantees, representing a trade-off between performance at physical boundaries and theoretical robustness

- \cite{odryFuzzyControlSelfbalancing2020} Odry et al.1 investigate an underactuated self-balancing platform characterized by a 3D-printed inverted pendulum structure and a low-cost sensing suite comprising a MEMS inertial measurement unit and incremental encoders23. The system is governed by an 8-dimensional state-space model derived from Euler-Lagrange equations, integrating motor dynamics into the global kinematic representation45. To ensure real-time performance on resource-limited 8-bit microcontrollers, the authors implement a decentralized control scheme featuring three parallel PD-type Mamdani fuzzy logic controllers, which are executed via Look-Up Tables (LUTs) to approximate the nonlinear control surface without significant computational latency6....
While the platform demonstrates stability against external disturbances through heuristic IF-THEN rules, the methodology assumes fixed inertial parameters and does not provide formal stability guarantees under parameter variation910. The reliance on a static fuzzy rule base and the lack of adaptive compensation for unmodeled friction or varying center-of-mass locations contrast with the requirements for robots operating under the dynamic uncertainties and payload fluctuations typical of human-robot interaction10.... Furthermore, the sensing setup utilizes a complementary filter for tilt estimation1314, which, while computationally efficient, may lack the robustness required for the complex acceleration profiles encountered during active human-robot collaboration.

\cite{wangHeadStabilizationWheeled2025} Wang et al.1 present a 6-degree-of-freedom (6-DOF) wheeled bipedal robot, the Diablo platform, utilizing a decoupled modeling architecture to address head stabilization on uneven terrain—a challenge often neglected by standard whole-body balancing controllers1.... The mechanical structure features serial kinematic legs with direct-drive joints, modeled as a Wheeled Spring-Damper Inverted Pendulum (W-SDIP) where the legs function as compliant elements to facilitate admittance shaping2.... A critical technical choice is the development of a proprioceptive-only force estimation module that avoids the mass and complexity of external force/torque sensors67. This estimator leverages a 2-DOF planar serial linkage model and the principle of virtual work to determine ground reaction forces, using a 2D acceleration-space logic to detect contact states4.... To maintain model fidelity despite simplifying wheel dynamics, the authors introduce empirical mapping coefficients to project wheel motor torques into the joint space, effectively treating omitted inertial properties as compensated disturbances910.
The control strategy adopts an adaptive admittance framework that regulates the robot's vertical trajectory by treating the head as a point mass subject to virtual spring-damper-inertia characteristics5.... This high-level controller dynamically modulates the reference leg length for a lower-level PD-plus-feedforward height regulator based on the estimated vertical contact force F 
z
 11.... While the admittance formulation is fundamentally comparable to strategies used in physical human-robot interaction for compliance, its application here focuses on world-frame stabilization to protect fragile payloads and maintain the accuracy of exteroceptive sensors6.... This approach provides a significant contrast to conventional wheeled inverted pendulum (WIP) systems that rely on fixed leg lengths or high-gain feedback, which typically propagate terrain-induced oscillations315. By achieving up to a 68.9\% reduction in mean absolute error for head height during terrain traversal, the framework demonstrates a robust mechanism for managing dynamic uncertainties and variable ground reaction forces without the need for specialized external sensing hardware

 - \cite{yuImpedanceControlIndustrial2025} Yu and Wang \cite{Yu2025} propose an integrated Model Predictive Control (MPC) and impedance control framework for industrial manipulators, validated on the ROCR-6 platform, to facilitate rapid and stable contact force establishment in unstructured environments12. The system is modeled using an Euler-Lagrange formulation, which is linearized through an acceleration-based control law and mapped to a discrete-time linear state-space representation using a first-order Euler discretization3.... A distinctive technical particularity of this approach is the linearization of the end-effector’s orientation deviation; by utilizing an axis-angle representation and performing a full rotation matrix transformation only for the initial step of the prediction horizon, the authors mitigate the computational burden of nonlinear orientation regulation1.... This formulation enables the use of convex quadratic optimization to handle explicit inequality constraints on force, velocity, and position, which are often absent in traditional second-order spring-damper impedance models7....
While the study addresses fully actuated manipulators, its methodology for handling interaction constraints is highly relevant to the safety requirements of human–robot interaction in underactuated self-balancing platforms1011. The control law demonstrates superior performance in reducing peak impact forces and achieving faster convergence to desired interaction forces compared to conventional impedance methods, which is critical when managing variable payloads and dynamic uncertainties1112. However, the underlying assumption that orientation variations remain small and away from singular configurations represents a significant design trade-off6. For a wheeled inverted pendulum, the reliance on such linearization may be more restrictive due to the larger angular excursions inherent in maintaining balance, suggesting that the "free evolution" and "state reachability" matrices used in this MPC formulation would require frequent re-computation or robustification to account for the platform’s unstable, nonlinear dynamics

- \cite{unluturkMachineLearningBased2022} Unluturk and Aydogdu \cite{Unluturk2022} investigate the control of an underactuated mobile inverted pendulum (U-MIP) using an integration of machine learning and adaptive fuzzy logic to manage discrete payload variations12. The mechanical structure comprises a two-wheeled platform with an actuation layout of dual PMDC motors and a non-standard load bar mounted on top to facilitate variable weights up to 495g23. The system's dynamic model is derived using a Lagrangian formulation and linearized about the vertical equilibrium, resting on the explicit assumptions of negligible motor inductance and zero wheel-floor friction4.... To counter the parametric uncertainties and center-of-gravity shifts induced by added mass, the authors propose an Adaptive Fuzzy Logic-Proportional Integral (AFL-PI) controller. A core technical particularity is the use of an Artificial Neural Network (ANN) as an online classifier that estimates the payload class—Low, Normal, or Heavy—by processing extracted features from the robot’s performance, specifically the mean absolute target tilt angle deviation error (MATTADE), linear displacement error (MATLDDE), and controller output deviation error (MATCODE)7....
This sensorless estimation approach offers a low-cost, robust alternative to physical load cells, mitigating risks associated with sensor failure in dynamic environments10. The control strategy employs a switching function (σ 
i
​
 ) to update fuzzy normalization (K 
1,i
​
 ,K 
2,i
​
 ) and denormalization (K 
3,i
​
 ) parameters in real-time, which experimentally improved body tilt angle performance by 55.62\% under heavy load conditions compared to non-adaptive fuzzy controllers11.... From a control-theoretic perspective, a significant trade-off is acknowledged: while the hybrid intelligent structure provides high practical robustness against unmodeled dynamics and environmental disturbances, it cannot analytically guarantee the global stability of the closed-loop system for nonlinear complex systems with uncertainties1415. This characteristic presents a point of contrast for human–robot interaction (HRI) scenarios, where formal stability guarantees are often required to ensure safety despite the controller's demonstrated precision in self-balancing and motion maneuvers

 - \cite{saltducajuModelBasedPredictiveImpedance2025} Salt Ducaju et al. propose a Model-Predictive Variable Impedance (PVI) controller for a Franka Emika Panda manipulator, modeling its Cartesian dynamics as a Linear Parameter-Varying (LPV) system under the assumption of a fully-actuated, nonredundant configuration1.... The framework integrates linearized Safety Control Barrier Functions (SCBFs) as inequality constraints within a linear Model Predictive Control (MPC) scheme to enforce operational space boundaries for safe physical human–robot interaction (pHRI)4.... By linearizing the typically quadratic forward-invariance conditions of the SCBFs, the authors derive a convex quadratic optimization problem that ensures global optimality and facilitates high-frequency execution at 1 kHz, significantly exceeding the sampling rates of traditional nonlinear MPC alternatives5.... Global asymptotic stability for both the nominal impedance behavior and the tracking error dynamics is formally guaranteed through a time-varying Lyapunov candidate9....
While the study focuses on a fixed-base manipulator, its predictive treatment of dynamic uncertainties is highly relevant to underactuated self-balancing platforms; specifically, the authors incorporate a linear model to estimate and predict external human-guidance forces based on Cartesian acceleration data12.... This predictive capability reduces trajectory error and the duration of risk-prone states during obstacle avoidance compared to standard 1-step SCBF formulations15.... However, the reliance on linearized dynamics and the assumption of linear human-guidance behavior present a critical trade-off: while the formulation gains computational efficiency and stability guarantees, its performance may degrade in scenarios involving highly nonlinear payload-platform couplings or abrupt, non-linearizable human interactions5.... For underactuated systems, such as wheeled inverted pendulums, the PVI framework offers a robust mechanism for enforcing safety envelopes, though the inherent underactuation would require extending the linearized SCBF constraints to account for the platform's non-holonomic and balancing constraints


- \cite{millsNonlinearModelPredictive2009} Mills et al. demonstrate the feasibility of real-time Nonlinear Model Predictive Control (NMPC) for an underactuated cart-pendulum system, which serves as a fundamental analog for self-balancing wheeled platforms12. The mechanical structure involves a single-input DC motor driving a belt-driven cart, where the state vector comprises cart position, pendulum angle, and their respective derivatives, estimated via finite difference approximations3.... The modeling assumes a discrete-time nonlinear state-space representation derived through Euler integration, specifically accounting for the center of mass without assuming it resides at the pendulum's geometric center6.... This approach addresses the system’s inherent instability and non-minimum phase characteristics while explicitly incorporating physical constraints on motor current and track length through a constrained optimization framework1....
The control strategy utilizes Sequential Quadratic Programming (SQP) to solve a non-convex optimization problem involving 61 variables and 241 constraints at a 40 Hz sampling rate10.... To maintain real-time performance on modest hardware, the authors employ a Gauss-Newton approximation of the Hessian of the Lagrangian and enforce a strict limit of four iterations per control cycle2.... While this methodology ensures responsiveness to manual disturbances and achieves autonomous swing-up, it relies on a deterministic model with fixed parameters, which contrasts with the requirements for platforms subject to variable payloads or the stochastic dynamic uncertainties inherent in human-robot interaction6.... The trade-off between optimization convergence and computational latency highlighted in this work provides a critical baseline for assessing the viability of NMPC in highly dynamic, constrained balancing applications17....
An analogy for this control approach is a high-speed chess player who only looks a few moves ahead and limits their thinking time for each turn; while they might not find the perfect strategy, their ability to react quickly allows them to stay in the game against a fast-moving opponent.

- \cite{zadOptimalControllerDesign2016} Zad et al.1 propose a continuous-time optimal Model Predictive Control (MPC) framework for a two-wheeled self-balancing robot, modeled fundamentally as an underactuated single inverted pendulum on a cart23. The system dynamics are linearized via small-angle approximations, and the state-space model is augmented to include an integral action for constant set-point tracking34. A core modeling assumption is the neglect of friction and the exclusion of the direct feedthrough term to satisfy the receding horizon principle25. The control strategy employs orthonormal functions to model the manipulated variable's trajectory, minimizing a quadratic cost function within a finite, moving optimization window to determine the optimal control law4....
While the authors demonstrate superior settling times and zero overshoot compared to a Ziegler-Nichols tuned PID controller, the study's relevance to dynamic human-robot interaction is constrained by its reliance on linearized plant models and the assumption of a static, point-mass payload2.... The robust performance is validated against a 10\% parameter perturbation, yet the framework lacks explicit adaptation mechanisms for the wide-scale mass and center-of-gravity shifts characteristic of variable payloads19. Furthermore, the formulation focuses on position tracking and stability under input constraints11, but does not explicitly address the safety or compliance requirements essential for direct human interaction.

- \cite{yajimaPostureStabilizationControl2023} Yajima et al.1 investigate an underactuated two-wheeled robotic platform featuring a high-mounted motorized arm for autonomous luggage transport, specifically addressing the longitudinal center of gravity (CoG) shifts inherent in asymmetric payload handling23. The system is modeled as a planar wheeled inverted pendulum (WIP) under a non-slip assumption, where the mechanical structure is characterized by three degrees of freedom controlled by only two wheel actuators2.... To mitigate the destabilizing effects of a variable CoG, the authors propose a cascaded control architecture: an outer-loop PI controller for position tracking is coupled with an inner-loop Lyapunov-based PD controller to ensure pitch stability6.... Robustness against modeling errors and interference is enhanced via a Synthesized Pitch Angle Disturbance Observer (SPADO), which estimates and compensates for integrated wheel and pitch disturbances910.
The core innovation lies in the integration of Repulsive Compliance Control (RCC) driven by a force-sensorless Reaction Torque Observer (RTOB)1112. Unlike traditional compliance control that follows external forces, RCC generates a pitch angle command in the opposite direction of the estimated payload reaction torque, effectively rebalancing the robot by shifting its posture to compensate for the steady-state CoG deviation6.... This methodology is highly relevant to platforms operating under dynamic uncertainties as it eliminates the need for explicit payload parameters while preventing the rapid longitudinal acceleration typically observed when WIP systems encounter unmodeled center of mass offsets6.... While the decoupled arm-body control structure simplifies implementation by utilizing observers to handle interference components, this design choice represents a trade-off that may necessitate careful tuning of the phase lead compensator to manage oscillations during high-dynamic transitions

- \cite{itoUnderactuatedControlTwoWheeled2023} Ito and Murakami propose an underactuated control strategy for a two-wheeled mobile robot equipped with an arm, utilizing torque constraint conditions and a disturbance observer to enhance stability and performance in the presence of external disturbances and payload variations1112.Ito and Murakami1 propose a robust control framework for a two-wheeled mobile robot (TWMR) configured as a double inverted pendulum with an integrated robotic arm, a design that facilitates payload handling in human-collaborative spaces but introduces significant dynamic instability12. The platform is inherently underactuated, where a single translational force input must simultaneously regulate the body’s pitch angle and longitudinal position23. To mitigate the modeling errors and disturbances inherent in variable payload scenarios, the authors implement a Predictive Functional Control (PFC) strategy augmented by a Synthesized Pitch Angle Disturbance Observer (SPADO) and Reaction Torque Observers (RTOB)45. The PFC formulation avoids the high computational cost of traditional Model Predictive Control by utilizing polynomial basis functions and coincidence points, while explicitly incorporating torque constraint conditions that prioritize pitch stabilization during actuator saturation6....
A critical technical feature of this architecture is the real-time update of the PFC’s internal model using RTOB-estimated torques to account for changes in mass and center of gravity (CoG)410. This method assumes a non-slip condition for the wheels and relies on a nominal inertia matrix tuned to specific arm configurations to maintain stability1112. By utilizing translational acceleration as a pseudo-input for disturbance compensation, the SPADO improves the precision of the internal model, allowing the system to outperform conventional cascade PI/PD controllers613. Such cascade architectures typically exhibit large translational errors and oscillatory responses due to the frequency separation requirements between the fast pitch-control loop and the slower position loop1415. Consequently, the integration of constrained predictive control with disturbance observers provides a more robust solution for self-balancing platforms subject to the dynamic uncertainties and strict safety constraints of human-robot interaction1...

- \cite{ahn2014}Ahn and Jung describe a two-wheeled mobile manipulator, the Balancing Service Robot (BSR), which features an underactuated inverted pendulum architecture enhanced by a height-adjustable linear waist and a unique separable modular structure. The modeling assumes the center of gravity is located on the wheel axis for the motion kinematics, though the sensing setup utilizes a complementary filter to fuse gyro and tilt sensor data to mitigate the limitations of cost-effective hardware. To stabilize the platform, a decoupled linear control scheme is implemented, utilizing a **PD controller for pitch regulation**—intentionally omitting integral action to prevent error accumulation from unknown center-of-gravity shifts—alongside PID controllers for position and heading. Dynamic uncertainties and unmodeled nonlinearities are further addressed through a **Radial Basis Function (RBF) neural network** that compensates for positional errors.
For human--robot interaction, the BSR adopts a **position-based impedance (admittance) strategy**, transforming external force inputs sensed at 6-DOF end-effectors into reference trajectory adjustments via a second-order admittance filter. This architecture facilitates cooperative tasks, such as door opening, by coupling the movements of the mobile base and the manipulators. However, a critical control-theoretic trade-off is identified: the **limited control bandwidth** of the system restricts its ability to maintain stability under high-frequency external disturbances or rapid force applications. This BSR framework is particularly relevant for systems operating under variable payloads, as it explicitly addresses the stability challenges posed by shifting centers of gravity and external interaction forces through a hybrid of linear regulation and adaptive compensation.

\cite{shiroma1996cooperative}
Shiroma et al. explore the cooperative transportation capabilities of a wheeled inverted pendulum (WIP), focusing on the integration of self-balancing with external force exertion. The platform utilizes a standard WIP configuration—driven by a DC motor with rate gyro and rotary encoder feedback—but incorporates a non-standard state-space formulation that treats the external interaction force $F_e$ as a state variable. To simplify the dynamics for control, the authors employ a linearized model under small-angle approximations and assume the interaction force is applied horizontally at the wheel axle height with static dynamics, i.e., $\dot{F}_e = 0$. The control architecture utilizes a pole-assignment state-feedback regulator augmented by an observer to simultaneously estimate the pendulum's inclination and the unknown external force.

A critical aspect of this work is the realization of "cooperative behavior" through a force control law that adds a feedforward compensation term to the stabilization loop. This enables the robot to follow human-initiated movement by responding to deviations between commanded and estimated interaction forces, effectively treating the object's impedance as a positive compliance. However, the reliance on a linearized model and the assumption of force application at the axle limit the system's robustness to complex dynamic uncertainties. Experimental results reveal a significant design trade-off: while the observer-based approach allows for stable interaction, low-frequency oscillations occur during transport because the friction compensation, tuned for no-load conditions, fails to accommodate the nonlinearities introduced by variable external loads. This highlights the limitations of linear state-space methods in high-fidelity human–robot interaction where payload-induced changes in system dynamics are substantial.

\cite{kim2025} Kim and Choi investigate a Longitudinally Extended Two-Wheeled Inverted Pendulum Robot (LE-TWIPR), which departs from traditional vertically oriented architectures to address stability and clearance challenges inherent in high-aspect-ratio horizontal platforms. The system utilizes a redundant sliding mechanism that actively modulates the longitudinal center of mass (CoM) via a ball-screw and pulley assembly, effectively compensating for asymmetric payloads that would otherwise exceed the torque capacity of standard wheel-based balancing. While the full nonlinear dynamics are derived through the Lagrangian method, the control-oriented model employs small-angle approximations to facilitate real-time optimization. A specialized sensing suite incorporating dual ultrasonic sensors at the chassis extremities enables a geometry-based slope estimation algorithm, which is critical for ensuring ground clearance in longitudinally extended frames.

The control architecture integrates Model Predictive Control (MPC) with a reduced-order disturbance observer (DOB) to mitigate effects from variable payloads and terrain irregularities. The MPC is formulated as a constrained quadratic program and solved using the OSQP solver, allowing the system to optimize wheel and slider trajectories while respecting state and input saturation limits. To handle dynamic uncertainties, the DOB estimates external forces in real time, which are then used to generate a slider position offset that neutralizes steady-state pitch errors—a significant improvement over conventional LQR controllers that often succumb to torso-ground contact under asymmetric loading. However, the slider reference generation relies on a stationary equilibrium assumption, which, while computationally efficient, limits the system's ability to adapt to payloads that vary dynamically during high-speed slope traversal. This framework is highly relevant to platforms operating under human–robot interaction constraints, as it demonstrates a robust method for managing significant center-of-mass shifts through active mechanical reconfiguration.

\cite{kawaharazuka2018} The TWIMP platform represents a non-standard hybrid approach that couples a musculoskeletal dual-arm upper body with a two-wheeled inverted pendulum base. Mechanically, the upper limb utilizes a tendon-driven architecture employing nonlinear elastic elements (O-rings) to facilitate compliant environmental contact and inherent impact resistance, while the lower limb provides high mobility through an in-wheel motor configuration. The control architecture is partitioned into a joint-space musculoskeletal controller, which utilizes an online-trained self-body image and muscle-tension optimization, and a postural balance controller based on a linear optimal regulator (LQR) derived from Lagrangian dynamics. Sensing is achieved through a redundant setup of motor-integrated sensors, external rotary encoders, and a trunk-mounted inertial measurement unit (IMU) to provide feedback for the state vector $[\theta, \phi, \dot{\theta}, \dot{\phi}]^T$.

To address dynamic uncertainties and center-of-mass (CoM) fluctuations during manipulation, the authors implement an adaptation law that modifies the reference pitch angle $\theta_{ref}$ proportional to wheel displacement. This approach is particularly relevant for human--robot interaction as the passive compliance of the musculoskeletal structure allows the robot to absorb significant external impacts—demonstrated through kicking and wall collisions—without structural damage. However, the design involves a critical trade-off: while the structural softness enhances safety and robustness during physical contact, it introduces significant modeling errors and transient oscillations that challenge the precision of standard state-feedback methods. This necessitates the gradual modification of reference states to prevent instability, highlighting the difficulty of maintaining high-bandwidth postural control in underactuated systems with soft, high-inertia upper bodies.

- \cite{seongheejeong2007} The I-PENTAR platform represents a sophisticated iteration of the wheeled inverted pendulum (WIP) architecture, integrating a dual-motor wheel drive system, a high-power waist joint, and 2-DOF arms to balance human-centric safety with high work capability. A critical design choice is the implementation of a double-motor method for each wheel to reduce backlash and increase power, while a rate gyro in the lower body provides the inclination feedback necessary for maintaining dynamic stability. The system is modeled as a 3D robot using Lagrangian dynamics that incorporate non-holonomic constraints, explicitly accounting for the coupling between the steering angle, inclination, and linear displacement. Control is achieved through a Linear Quadratic Regulator (LQR) derived from a state-space representation linearized about the upright equilibrium, assuming small perturbations for the balancing, running, and steering tasks.

While the platform successfully executes complex transition maneuvers—such as standing and sitting by utilizing arm-mounted rollers to shift between static and dynamic stability—the control synthesis relies on a fixed linearized model. This dependence on a fixed-gain linear controller highlights a technical trade-off: while it ensures stable fundamental motions, it lacks the inherent robustness or adaptation required to handle the variable payloads and center-of-gravity (CoG) shifts characteristic of intensive human-assistant tasks. However, I-PENTAR’s mechanical inclusion of a waist joint for intentional mass redistribution provides a relevant physical framework for contrasting traditional optimal control with more advanced robust or learning-based strategies designed to mitigate dynamic uncertainties in underactuated platforms.

**Analogy:** Managing the I-PENTAR's balance using fixed LQR control while it performs work is much like a tightrope walker who can perfectly maintain their center of gravity in a calm environment, but might struggle if they suddenly had to carry a heavy, shifting object without changing their fundamental balancing technique.


\cite{abeygunawardhana2010} Abeygunawardhana et al. investigate the stabilization of a two-wheel mobile manipulator characterized by a coaxial wheel system and a three-link manipulator, totaling four degrees of freedom,. The mechanical architecture incorporates a passive joint at the wheel axis, rendering the platform underactuated and requiring posture stabilization via wheel torque,. To manage the system's high dimensionality, the authors utilize a center of gravity (COG) workspace control approach that simplifies the 4-DOF kinematics into a virtual double inverted pendulum (DIP) model,,. This modeling assumption relies on lumped mass representations at the link tips and a sensing suite comprising motor encoders and a gyro sensor for the unactuated body pitch,,.

The control synthesis employs a second-order sliding mode control (SMC) formulation using a twisting algorithm designed to achieve finite-time convergence to the sliding manifold while mitigating the chattering phenomenon,,. To address the dynamic uncertainties and external torques inherent in mobile manipulation, the authors integrate a disturbance observer (DOB) that compensates for model mismatches and interactive forces,. This hybrid strategy ensures exponential stability of the passive joint error by treating external disturbances as a lumped term, which is particularly relevant for platforms operating under variable payloads,,. A significant design trade-off noted is the necessity of low-pass filtering and pseudo-differentiators to estimate acceleration, which introduces potential phase lag in exchange for reduced noise sensitivity in the DOB loop,.

\cite{kanazawa2023} Kanazawa et al. investigate a two-wheeled robot with an arm (TWRwA), a three-degree-of-freedom underactuated system where front-back center of gravity (CoG) shifts significantly impact wheel position control performance,,. Unlike standard wheeled inverted pendulums, this platform incorporates an active arm whose varying posture and payload create substantial dynamic uncertainties,. The researchers employ a cascaded control architecture where an outer-loop PI wheel position controller provides a reference for an inner-loop pitch stabilization controller,. This inner loop is formulated using the **Lyapunov stability theorem** and integrated with a **Synthesized Pitch Angle Disturbance Observer (SPADO)** to linearize the nonlinear dynamics and reject ground-induced disturbances,,. 

To address variable payloads, the authors propose a **model-based CoG compensator** that utilizes a **Reaction Torque Observer (RTOB)** on the arm motor to estimate load-derived torque,. This approach statically calculates a desired pitch angle to maintain the total CoG directly above the wheel axle, facilitating faster convergence compared to conventional Repulsive Compliance Control (RCC) methods,,. While the static determination of the pitch angle improves wheel position tracking during load transitions, the strategy relies on **small-angle approximations** and the prior identification of arm friction and gravity terms, which may limit its robustness during high-dynamic maneuvers or complex human-robot interaction scenarios,. This model-based compensation offers a technically precise alternative to impedance-based methods, emphasizing the trade-off between fast, static convergence and the flexibility of dynamic compliance,.


