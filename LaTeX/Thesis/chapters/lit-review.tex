\begin{comment}
\section{State of the Art}

This section reviews existing literature relevant to the development of control frameworks for self-balancing robotic platforms, with particular emphasis on two-wheeled inverted pendulum systems. The review examines model predictive control strategies, disturbance observation techniques, adaptive control approaches, and architectures for physical human-robot interaction, highlighting the evolution of methodologies and identifying areas where further advancement is necessary.

\subsection{Model Predictive Control for Wheeled Inverted Pendulum Systems}

Model Predictive Control has emerged as a powerful framework for stabilizing underactuated wheeled inverted pendulum robots due to its capacity to handle system constraints and predict future behavior over finite horizons. Early implementations demonstrated the feasibility of real-time nonlinear MPC using Sequential Quadratic Programming to solve constrained optimization problems \cite{millsNonlinearModelPredictive2009}. The approach employed discrete-time nonlinear state-space representations with Gauss-Newton Hessian approximations to maintain 40 Hz sampling rates, successfully achieving autonomous swing-up and manual disturbance rejection. However, reliance on deterministic models with fixed parameters limited applicability in scenarios with stochastic uncertainties.

Comparative evaluations between Linear Quadratic Regulator and Model Predictive Control strategies have demonstrated MPC's superior performance in settling time, overshoot reduction, and control effort optimization \cite{mishraControlTwoWheelSelfBalancing2024}. Despite these advantages, such implementations typically assumed static inertial parameters and evaluated robustness through impulse disturbances rather than parametric variations, leaving questions regarding performance under continuous payload changes.

Continuous-time optimal MPC frameworks employing orthonormal functions for control trajectory modeling have shown improvements over classical PID controllers \cite{zadOptimalControllerDesign2016}, yet continued to rely on linearized plant models and static payload assumptions. The absence of explicit adaptation mechanisms for mass and center-of-gravity shifts represents a persistent limitation across these formulations.

Advanced implementations integrating MPC with reduced-order disturbance observers have addressed variable payloads and terrain irregularities through constrained quadratic programming \cite{kim2025}. The generation of corrective actuator offsets based on disturbance estimation demonstrated improvements over conventional linear quadratic regulators under asymmetric loading. However, reliance on stationary equilibrium assumptions constrained adaptability during highly dynamic operations, particularly in scenarios involving simultaneous payload manipulation and mobility.

\subsection{Adaptive Model Predictive Control and Parameter-Varying Formulations}

The challenge of discrete payload variations has motivated adaptive MPC strategies that explicitly update predictive models in response to changing system parameters. Dual-loop architectures integrating Adaptive Model Predictive Control with Linear Parameter Varying formulations have employed scheduling of local linear models by varying mass parameters \cite{onkolAdaptiveModelPredictive2018}. Time-Varying Kalman Filters maintained estimator consistency as dynamics shifted, demonstrating superior performance compared to PID and feedback linearization approaches under discrete mass changes representative of pick-and-place operations. The virtual link abstraction used to simplify multi-link manipulator dynamics, while computationally efficient, assumed sufficiently fast internal dynamics that may not hold during complex payload manipulation tasks.

Alternative formulations utilizing Predictive Functional Control with polynomial basis functions and coincidence points have reduced computational burden compared to traditional MPC \cite{itoUnderactuatedControlTwoWheeled2023}. Explicit incorporation of torque constraint conditions prioritizing pitch stabilization during actuator saturation, combined with real-time internal model updates based on reaction torque estimation, enabled robust performance under variable payloads. This approach outperformed conventional cascade controllers that exhibited large translational errors due to frequency separation requirements. Nevertheless, the methodology required careful tuning of phase compensation and relied on specific assumptions about arm configurations, suggesting potential limitations in generalization across diverse manipulation scenarios.

The gap between these adaptive formulations and the requirements of continuous payload manipulation becomes apparent when considering platforms that must simultaneously maintain balance, track trajectories, and respond to smoothly varying loads during active human collaboration. Existing approaches typically address either discrete payload changes or require offline parameter identification, leaving open the question of seamless adaptation during ongoing manipulation tasks.

\subsection{Disturbance Observation and Real-Time Parameter Estimation}

Disturbance observers have become integral to robust control of self-balancing platforms, providing estimation and compensation for modeling errors, external forces, and parameter variations. The Synthesized Pitch Angle Disturbance Observer estimates integrated wheel and pitch disturbances, enhancing robustness when integrated with cascaded control architectures \cite{yajimaPostureStabilizationControl2023}. This methodology effectively mitigates destabilizing effects from variable center-of-gravity shifts in systems with fixed control structures.

Reaction Torque Observer techniques enable force-sensorless estimation of payload-induced torques through motor current monitoring and inverse dynamics models \cite{yajimaPostureStabilizationControl2023, kanazawa2023}. For platforms with active arms or fork mechanisms, estimation of load-derived torques facilitates compensation strategies that rebalance robots through postural reference adjustment. The elimination of physical force sensors reduces system complexity while maintaining sufficient accuracy for steady-state compensation, though careful tuning remains necessary to manage oscillations during high-dynamic transitions.

Integration of multiple disturbance observers within predictive control frameworks has demonstrated enhanced capability to handle dynamic uncertainties \cite{itoUnderactuatedControlTwoWheeled2023}. The combination of Synthesized Pitch Angle Disturbance Observer with Reaction Torque Observers provides comprehensive compensation for ground-induced disturbances and payload variations, enabling real-time internal model updates. By utilizing translational acceleration as pseudo-input for disturbance compensation, such architectures improve internal model precision beyond conventional cascade controllers.

Despite these advances, the integration of disturbance observation specifically with nonlinear model predictive control for real-time parameter adaptation remains underexplored. Most implementations combine observers with fixed-structure controllers or utilize observer outputs for discrete model switching rather than continuous predictive model updates. This represents a significant opportunity for enhancing both robustness and performance in platforms subject to continuous payload variations.

\subsection{Adaptive Control Strategies for Variable Payloads}

Various adaptive control strategies have been developed to manage variable payloads and uncertain dynamic parameters. Adaptive Nonlinear Proportional-Derivative control schemes utilizing time-varying, nonlinear gains that adapt based on tracking error magnitude have demonstrated superior external impulse rejection compared to fixed-gain controllers \cite{nguyenAdaptiveNonlinearPD2024}. Genetic Algorithm-based offline parameter optimization addresses design trade-offs between transient response and tuning effort, though formulations typically assume nominal kinematic parameters and may exhibit limitations under significant parameter drift.

Neural network integration with adaptive control architectures has provided mechanisms for online estimation of unknown dynamic parameters \cite{nghiaAdaptiveNeuralSliding2022}. Radial basis function networks effectively estimate centrifugal, Coriolis, and gravitational forces with Lyapunov-proven stability, though such approaches typically assume relatively static mechanical configurations and do not explicitly account for time-varying center-of-mass shifts during active manipulation.

Model Reference Adaptive Control strategies employing Lyapunov-based adaptation mechanisms have addressed nonlinearities and environmental disturbances through continuous gain parameter updates \cite{isdaryaniDesignImplementationTwoWheeled}. While effective for balancing stability under varying loads, reliance on simplified lower-order reference models and omission of certain state couplings may present limitations where precise positioning during payload manipulation is critical.

Machine learning-based payload classification integrated with adaptive fuzzy logic control has enabled discrete payload management without physical load sensing \cite{unluturkMachineLearningBased2022}. Artificial neural networks operating as online classifiers estimate payload categories through performance metrics, enabling real-time fuzzy controller parameter updates. However, the hybrid intelligent structure cannot analytically guarantee global stability for nonlinear systems with uncertainties, contrasting with formal stability requirements for safety-critical human-robot interaction.

The common limitation across these adaptive strategies is their focus on either discrete payload classification or slow-varying parameter adaptation, rather than seamless integration with predictive control frameworks capable of anticipating the effects of payload changes on future system behavior.

\subsection{Physical Human-Robot Interaction and Compliance Control}

Physical human-robot collaboration requirements have motivated control architectures prioritizing safety, compliance, and adaptability to human-induced forces. Position-based admittance strategies transform external force inputs into reference trajectory adjustments via second-order admittance filters, facilitating cooperative tasks through coupled mobile base and manipulator movements \cite{ahn2014}. However, limited control bandwidth restricts stability maintenance under high-frequency external disturbances, highlighting trade-offs between compliance and robust stabilization.

Repulsive Compliance Control represents an alternative paradigm where control commands oppose estimated payload reaction torques, rebalancing robots by shifting posture to compensate for steady-state center-of-gravity deviations \cite{yajimaPostureStabilizationControl2023}. Unlike traditional compliance control that follows external forces, this approach prevents rapid longitudinal acceleration when encountering unmodeled center-of-mass offsets, eliminating explicit payload parameter requirements while maintaining stability.

Model-based center-of-gravity compensation utilizing reaction torque observation has been proposed for platforms with active arms, where static calculation of desired pitch angles maintains total center of gravity above wheel axles \cite{kanazawa2023}. This approach facilitates faster convergence compared to dynamic compliance methods, though reliance on small-angle approximations and prior identification of arm friction and gravity terms may limit robustness during high-dynamic maneuvers.

Integration of Model Predictive Control with impedance frameworks through Linear Parameter-Varying formulations has incorporated Safety Control Barrier Functions as inequality constraints \cite{saltducajuModelBasedPredictiveImpedance2025}. Linearization of forward-invariance conditions enables convex quadratic optimization with high-frequency execution, while predictive estimation of external human-guidance forces based on Cartesian acceleration reduces trajectory errors during obstacle avoidance. However, assumptions of linearized dynamics and linear human-guidance behavior present limitations in scenarios involving highly nonlinear payload-platform couplings or abrupt human interactions.

For manipulation tasks requiring rapid force establishment, integrated MPC and impedance control frameworks have been developed with explicit inequality constraints on force, velocity, and position \cite{yuImpedanceControlIndustrial2025}. Such approaches demonstrate superior performance in reducing peak impact forces and achieving faster convergence to desired interaction forces, though underlying assumptions of small orientation variations may be restrictive for platforms experiencing large angular excursions inherent in maintaining balance.

The challenge of detecting and responding to unexpected human contact, particularly in the presence of varying payloads that alter system response characteristics, remains an area requiring further investigation. Most existing approaches assume either known interaction patterns or operate in environments where collision events can be distinguished from normal operations through simple threshold-based methods.

\subsection{Collision Detection and Safety-Critical Control}

Observer-based collision detection methods have been explored primarily for fully actuated manipulators, where external force estimation through motor current monitoring enables identification of unexpected contacts. For underactuated self-balancing platforms, the challenge is compounded by the need to distinguish genuine collisions from the normal disturbances caused by balancing dynamics and payload manipulation.

Proprioceptive-only force estimation modules avoiding external force-torque sensors have been developed using principles of virtual work and planar serial linkage models \cite{wangHeadStabilizationWheeled2025}. Integration with adaptive admittance frameworks dynamically modulating reference trajectories based on estimated forces achieves significant oscillation reductions during terrain traversal. However, these methods focus on continuous environmental interactions rather than discrete collision events requiring immediate protective responses.

The gap in existing literature concerns the development of collision detection mechanisms that leverage predictive control frameworks themselves as sources of diagnostic information. The prediction errors inherent in model predictive control, when properly analyzed, may provide rich signals for identifying anomalous behaviors that deviate from expected system dynamics, even in the presence of varying payloads and ongoing manipulation tasks.

\subsection{Cooperative Transportation and Force Control}

Integration of self-balancing with external force exertion capabilities has been explored for cooperative transportation applications \cite{shiroma1996cooperative}. State-space formulations treating external interaction forces as state variables, combined with observer-based estimation, enable force control laws facilitating human-initiated movement. However, reliance on linearized models and assumptions of force application at specific locations limit robustness to complex dynamic uncertainties. Low-frequency oscillations during transport arising from friction compensation tuned for no-load conditions highlight limitations of linear state-space methods when payload-induced changes are substantial.

\subsection{Summary and Positioning}

The literature demonstrates substantial progress in control strategies for wheeled inverted pendulum robots, with Model Predictive Control emerging as a particularly capable framework due to its constraint-handling and prediction capabilities. Disturbance observers have proven effective for parameter estimation and external force compensation, while various adaptive control strategies have addressed payload variations through discrete classification or slow parameter updates.

However, several critical challenges remain unresolved. Most MPC implementations for self-balancing platforms rely on fixed-parameter models or discrete adaptation schemes that do not seamlessly handle continuous payload variations during active manipulation. While disturbance observers provide effective compensation in isolation, their integration with nonlinear predictive control for real-time model updates during simultaneous balancing and payload handling is underexplored. Furthermore, collision detection mechanisms capable of operating reliably in the presence of varying payloads and distinguishing genuine collisions from normal operational disturbances represent an open research question.

The opportunity exists to develop integrated control frameworks that combine the prediction and constraint-handling capabilities of nonlinear MPC with the real-time parameter estimation of disturbance observers, specifically addressing platforms with active payload manipulation mechanisms such as fork lifts or arms. Such frameworks must operate reliably in human-centric environments where unexpected interactions may occur, requiring sophisticated collision detection and response strategies that leverage the predictive nature of the control architecture itself. This represents the focus of the present work.

\section{Previous Work}

The research presented in this thesis builds upon previous work conducted within the same research group on the control of self-balancing wheeled robotic platforms. In particular, a closely related study was carried out by a former senior student of the author, whose thesis investigated the stabilization and control of a two-wheeled inverted pendulum robot operating under nominal conditions.

In that work, the primary focus was placed on the modeling and control of a self-balancing mobile robot with a fixed mechanical configuration and limited consideration of payload variability. Classical and model-based control strategies were employed to achieve stable balancing and trajectory tracking, demonstrating the feasibility of reliable control for underactuated wheeled inverted pendulum systems in structured environments. The implementation included disturbance observers and PD controllers, with successful validation in simulation environments. Mass estimation techniques were developed and trajectory tracking in both horizontal and vertical axes was demonstrated.

While this prior research established a solid foundation for the stabilization of self-balancing robotic platforms, several limitations were identified. The control strategies did not explicitly address human-robot interaction scenarios, validation was restricted to simulation without experimental verification on physical hardware, and the rigidity of the linear actuator mechanism resulted in limited effectiveness of the observer-based mass estimation when tested on the real robot. Furthermore, the fixed-parameter control approach did not account for dynamically changing system parameters or unexpected interactions with humans, factors that are critical in practical human-robot collaboration scenarios where payload properties may be unknown or time-varying and external disturbances can significantly affect system behavior.

The present thesis extends this earlier work by explicitly considering variable payload conditions and dynamic uncertainties within the control design. By building upon the insights and experimental framework developed by the former senior student, this research advances toward more robust, adaptive, and safety-oriented control strategies suitable for real-world human-centric environments.
\end{comment}

\section{State of the Art}

This section reviews the literature relevant to the development of control frameworks for self-balancing robotic platforms, with particular emphasis on two-wheeled inverted pendulum systems. The discussion covers model predictive control strategies, disturbance observation techniques, adaptive control approaches, and architectures for physical human--robot interaction. Emphasis is placed on the evolution of these methodologies and on identifying limitations that motivate further research.

\subsection{Model Predictive Control for Wheeled Inverted Pendulum Systems}

Model Predictive Control (MPC) has emerged as a powerful framework for stabilizing underactuated wheeled inverted pendulum robots due to its ability to explicitly handle system constraints and predict future behavior over finite horizons. Early implementations demonstrated the feasibility of real-time nonlinear MPC using Sequential Quadratic Programming to solve constrained optimization problems \cite{millsNonlinearModelPredictive2009}. These approaches employed discrete-time nonlinear state-space models and Gauss--Newton Hessian approximations to achieve sampling rates on the order of 40~Hz, enabling autonomous swing-up and effective rejection of manual disturbances. Nevertheless, the reliance on deterministic models with fixed parameters limited robustness in the presence of stochastic uncertainties.

Comparative studies between Linear Quadratic Regulator and MPC strategies have highlighted the superior performance of MPC in terms of settling time, overshoot reduction, and control effort optimization \cite{mishraControlTwoWheelSelfBalancing2024}. Despite these advantages, such implementations typically assumed constant inertial parameters and assessed robustness primarily through impulse disturbances rather than continuous parametric variations. As a result, system behavior under sustained payload changes remains insufficiently characterized.

Continuous-time optimal MPC formulations based on orthonormal function parameterizations of control trajectories have demonstrated improvements over classical PID controllers \cite{zadOptimalControllerDesign2016}. However, these approaches continued to rely on linearized plant models and static payload assumptions. The absence of explicit mechanisms to accommodate mass and center-of-gravity variations represents a recurring limitation.

More recent work integrating MPC with reduced-order disturbance observers has addressed variable payloads and terrain irregularities through constrained quadratic programming \cite{kim2025}. By generating corrective actuator offsets based on estimated disturbances, these methods outperformed conventional linear quadratic regulators under asymmetric loading conditions. Nonetheless, their dependence on stationary equilibrium assumptions limited adaptability during highly dynamic operations involving simultaneous mobility and payload manipulation.

\subsection{Adaptive Model Predictive Control and Parameter-Varying Formulations}

Discrete payload variations have motivated the development of adaptive MPC strategies that explicitly update predictive models in response to changing system parameters. Dual-loop architectures combining Adaptive MPC with Linear Parameter-Varying formulations have employed mass-dependent scheduling of local linear models \cite{onkolAdaptiveModelPredictive2018}. Time-varying Kalman filters ensured estimator consistency as system dynamics evolved, resulting in improved performance relative to PID and feedback linearization approaches under representative pick-and-place payload changes. However, the virtual link abstractions used to simplify multi-link dynamics assumed sufficiently fast internal dynamics, which may not hold during complex manipulation tasks.

Alternative formulations based on Predictive Functional Control using polynomial basis functions and coincidence points have reduced computational complexity compared to classical MPC \cite{itoUnderactuatedControlTwoWheeled2023}. Explicit torque constraint handling prioritized pitch stabilization during actuator saturation, while real-time internal model updates based on reaction torque estimation enabled robust operation under varying payloads. Although this approach outperformed conventional cascade controllers, it required careful phase compensation tuning and relied on specific assumptions regarding arm configurations, potentially limiting generalization.

These limitations become more pronounced in scenarios involving continuous payload manipulation, where platforms must simultaneously maintain balance, track trajectories, and respond to smoothly varying loads during physical human--robot interaction. Existing adaptive MPC formulations generally address discrete parameter changes or rely on offline identification, leaving seamless online adaptation during ongoing manipulation tasks largely unexplored.

\subsection{Disturbance Observation and Real-Time Parameter Estimation}

Disturbance observers play a central role in enhancing robustness for self-balancing platforms by estimating modeling errors, external forces, and parameter variations. The Synthesized Pitch Angle Disturbance Observer estimates integrated wheel and pitch disturbances, improving stability when combined with cascaded control architectures \cite{yajimaPostureStabilizationControl2023}. This approach effectively mitigates destabilizing effects arising from center-of-gravity shifts within fixed control structures.

Reaction Torque Observer techniques enable sensorless estimation of payload-induced torques through motor current measurements and inverse dynamics models \cite{yajimaPostureStabilizationControl2023, kanazawa2023}. For platforms equipped with active arms or lifting mechanisms, such estimations support compensation strategies based on postural reference adjustment. While eliminating physical force sensors reduces system complexity, careful tuning is required to prevent oscillations during highly dynamic transitions.

The integration of multiple disturbance observers within predictive control frameworks has further enhanced robustness to dynamic uncertainties \cite{itoUnderactuatedControlTwoWheeled2023}. Combining pitch angle and reaction torque observers enables compensation for both ground-induced disturbances and payload variations, facilitating real-time internal model refinement. Despite these advances, the systematic integration of disturbance observation with nonlinear MPC for continuous parameter adaptation remains limited. Observer outputs are often used for fixed compensation or discrete model switching rather than direct updates of predictive models.

\subsection{Adaptive Control Strategies for Variable Payloads}

Beyond predictive control, various adaptive strategies have been proposed to handle variable payloads and uncertain dynamics. Adaptive nonlinear proportional--derivative controllers with time-varying gains have demonstrated improved rejection of external impulses compared to fixed-gain schemes \cite{nguyenAdaptiveNonlinearPD2024}. Offline parameter optimization using genetic algorithms reduces tuning effort but generally assumes nominal system parameters, limiting effectiveness under significant parameter drift.

Neural-network-based adaptive controllers provide online estimation of unknown dynamics, including centrifugal, Coriolis, and gravitational effects \cite{nghiaAdaptiveNeuralSliding2022}. While Lyapunov-based formulations ensure stability, these approaches typically assume slowly varying mechanical configurations and do not explicitly address time-varying center-of-mass shifts during manipulation.

Model Reference Adaptive Control schemes have also been applied to self-balancing robots, employing Lyapunov-based adaptation laws to handle nonlinearities and disturbances \cite{isdaryaniDesignImplementationTwoWheeled}. Although effective for balance stabilization, the use of simplified reference models and reduced-order dynamics may limit positioning accuracy during payload handling.

Machine learning approaches integrating payload classification with adaptive fuzzy control have enabled discrete payload management without physical load sensors \cite{unluturkMachineLearningBased2022}. However, the lack of analytical stability guarantees limits applicability in safety-critical human--robot interaction contexts. Overall, these adaptive strategies tend to focus on discrete classification or slow parameter adaptation rather than continuous integration with predictive control.

\subsection{Physical Human--Robot Interaction and Compliance Control}

Physical human--robot collaboration has driven the development of control architectures emphasizing safety, compliance, and adaptability. Position-based admittance control transforms external forces into reference trajectory adjustments via second-order filters, facilitating cooperative tasks involving mobile bases and manipulators \cite{ahn2014}. However, limited control bandwidth introduces trade-offs between compliance and stabilization under high-frequency disturbances.

Repulsive Compliance Control offers an alternative approach by opposing estimated payload reaction torques to rebalance the robot through postural adjustments \cite{yajimaPostureStabilizationControl2023}. This strategy prevents excessive longitudinal acceleration caused by unmodeled center-of-mass offsets without requiring explicit payload parameter identification.

Model-based center-of-gravity compensation using reaction torque observation has also been proposed for platforms with active arms \cite{kanazawa2023}. By statically computing pitch references that align the total center of gravity above the wheel axle, faster convergence is achieved compared to dynamic compliance methods. Nevertheless, reliance on small-angle assumptions and prior parameter identification may reduce robustness during aggressive maneuvers.

The integration of MPC with impedance control through Linear Parameter-Varying formulations has introduced Safety Control Barrier Functions as inequality constraints \cite{saltducajuModelBasedPredictiveImpedance2025}. While enabling high-frequency convex optimization, these approaches typically rely on linearized dynamics and simplified human interaction models. Similar limitations arise in MPC-based force control frameworks for manipulation tasks \cite{yuImpedanceControlIndustrial2025}, where small orientation assumptions restrict applicability to strongly nonlinear balancing scenarios.

\subsection{Collision Detection and Safety-Critical Control}

Collision detection for self-balancing platforms presents unique challenges due to underactuation and the presence of balancing-induced disturbances. Observer-based force estimation methods developed for fully actuated manipulators have limited direct applicability in this context.

Proprioceptive force estimation techniques based on virtual work principles and planar linkage models have been proposed to avoid external force sensors \cite{wangHeadStabilizationWheeled2025}. When combined with adaptive admittance control, these methods reduce oscillations during continuous environmental interaction. However, they are primarily designed for sustained contact rather than discrete collision events requiring rapid protective responses.

An open research gap lies in leveraging predictive control frameworks themselves for collision detection. Deviations between predicted and measured system behavior within MPC formulations may provide informative indicators of anomalous events, even under varying payload conditions and active manipulation.

\subsection{Cooperative Transportation and Force Control}

Self-balancing platforms capable of exerting controlled external forces have been explored in cooperative transportation scenarios \cite{shiroma1996cooperative}. State-space formulations treating interaction forces as estimated states enable human-initiated motion through observer-based force control. Nonetheless, linearized models and assumptions regarding force application points limit robustness, particularly under significant payload-induced parameter changes. Low-frequency oscillations observed during transport further highlight the limitations of fixed-parameter linear models.

\subsection{Summary and Positioning}

The reviewed literature demonstrates significant progress in the control of wheeled inverted pendulum robots, with MPC emerging as a particularly effective framework due to its predictive and constraint-handling capabilities. Disturbance observers and adaptive control strategies have enhanced robustness to payload variations and external forces.

However, key challenges remain unresolved. Most MPC-based approaches rely on fixed-parameter models or discrete adaptation schemes that do not accommodate continuous payload variations during active manipulation. The integration of disturbance observers with nonlinear MPC for real-time model adaptation remains limited, and reliable collision detection under varying payload conditions is still an open problem.

These gaps motivate the development of integrated control frameworks that combine nonlinear MPC with real-time disturbance-based parameter estimation, specifically targeting self-balancing platforms with active payload manipulation capabilities. Such frameworks must operate robustly in human-centric environments and exploit the predictive structure of MPC for enhanced safety and interaction awareness. This objective defines the scope of the present work.

\section{Previous Work}

The research presented in this thesis builds upon prior work conducted within the same research group on the control of self-balancing wheeled robotic platforms. Earlier studies investigated the stabilization and trajectory tracking of two-wheeled inverted pendulum robots operating under nominal conditions and fixed mechanical configurations.

In this prior work, classical and model-based control strategies were employed to achieve stable balancing and trajectory tracking in structured environments. Disturbance observers and proportional--derivative controllers were implemented and validated in simulation, demonstrating the feasibility of reliable control for underactuated systems. Mass estimation techniques were developed, and trajectory tracking in both horizontal and vertical directions was successfully demonstrated.

Despite establishing a solid foundation, several limitations were identified. Human--robot interaction scenarios were not explicitly considered, validation was restricted to simulation without experimental verification on physical hardware, and the rigidity of the linear actuator mechanism reduced the effectiveness of observer-based mass estimation when applied to the real platform. Moreover, fixed-parameter control strategies did not account for dynamically changing system parameters or unexpected human interactions.

The present thesis extends this earlier work by explicitly addressing variable payload conditions and dynamic uncertainties within the control design. Building upon the established modeling and control framework, the research advances toward more adaptive, robust, and safety-oriented strategies suitable for real-world human-centric environments.
