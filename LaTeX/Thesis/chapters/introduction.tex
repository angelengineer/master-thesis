In recent years, modern industrial environments have experienced a growing integration of robotic systems designed to operate in close proximity to humans. This trend is driven by demographic challenges such as declining birth rates and labor shortages, as well as by the suitability of robots for repetitive, physically demanding, or ergonomically harmful tasks. By delegating such activities to robotic platforms, human workers can focus on roles that require creativity, decision-making, and higher-level cognitive skills.

The expanding deployment of robots in human-centric environments has increased the demand for mobile platforms capable of operating safely and reliably under dynamic and uncertain conditions. As robotic systems move beyond isolated industrial cells into shared workspaces, homes, and public environments, physical human--robot collaboration becomes a central requirement. In these scenarios, robots must not only navigate and manipulate objects effectively but also adapt to unpredictable interactions with humans and changes in the surrounding environment. Within this context, ensuring stable and robust control of mobile robotic platforms subject to external disturbances and varying payloads is a fundamental challenge, particularly when safe physical interaction is required.

Self-balancing wheeled robots represent an inherently unstable class of mobile platforms that offer notable advantages over statically stable alternatives, including improved maneuverability, higher energy efficiency, and enhanced accessibility in constrained environments. However, these benefits come at the cost of increased control complexity due to their underactuated nature, strongly nonlinear dynamics, and sensitivity to external perturbations and payload variations. When such platforms are intended for physical human--robot collaboration, stable operation under uncertain dynamic conditions becomes essential to guarantee safety and reliability, especially during tasks involving load transportation.

This thesis focuses on the design, analysis, and implementation of a robust control framework for a two-wheeled differential-drive inverted pendulum robot equipped with a vertically actuated load manipulation mechanism, referred to as the Two-Wheeled Forklift Robot (TWFR), developed at the Keio Murakami Laboratory. The platform consists of a main chassis located at the wheel axis and a prismatic actuator that extends and retracts a fork-like end effector for payload handling. This configuration allows the robot to transport objects with varying mass and inertial properties, reflecting realistic use cases in collaborative logistics, personal assistance, and shared workspace scenarios where close physical interaction between humans and robots is required.

The control problem is further complicated by the platform’s limited sensing capabilities. Available measurements are restricted to wheel encoders, a linear actuator position sensor, an angular sensor measuring the orientation of the fork joint, and an inertial measurement unit (IMU) providing the pitch angle. The resulting underactuated system dynamics, combined with time-varying parameters induced by payload changes and geometric reconfiguration during fork manipulation, pose significant challenges for controller design. Additionally, the control system must ensure robust performance in the presence of modeling uncertainties, external disturbances such as surface irregularities and human-induced forces, and abrupt dynamic changes caused by collisions or sudden load variations, while maintaining stability and operational safety during physical human--robot interaction.

This work contributes to the field of mobile robotics by proposing a comprehensive control methodology that explicitly addresses the challenges associated with self-balancing platforms operating under variable payload conditions and dynamic uncertainties in human-shared environments. The proposed approach integrates rigorous theoretical analysis with experimental validation, aiming to enhance system stability, adaptability, and safety in physical human--robot collaboration scenarios, thereby facilitating the deployment of such platforms in real-world applications.

The remainder of this thesis is organized as follows. Chapter~2 reviews the relevant literature on wheeled inverted pendulum robots, with particular emphasis on control strategies and their limitations in collaborative settings. Chapter~3 defines the objectives of this research, outlining the performance criteria and constraints of the control system. Chapter~4 presents the overall methodology adopted in this work. Chapter~5 details the system modeling and derivation of the dynamic equations used for control design. Chapter~6 describes the implemented control strategies, including nonlinear model predictive control and disturbance observer techniques. Chapter~7 presents the simulation setup and results, while Chapter~8 discusses the obtained findings and their implications. Finally, Chapter~9 concludes the thesis and outlines directions for future research.