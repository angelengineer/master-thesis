\section{Previous Work}

The research presented in this thesis builds upon previous work conducted within the same research group on the control of self-balancing wheeled robotic platforms. In particular, a closely related study was carried out by a former senior student of the author, whose thesis investigated the stabilization and control of a two-wheeled inverted pendulum robot operating under nominal conditions.

In that work, the primary focus was placed on the modeling and control of a self-balancing mobile robot with a fixed mechanical configuration and limited consideration of payload variability. Classical and model-based control strategies were employed to achieve stable balancing and trajectory tracking, demonstrating the feasibility of reliable control for underactuated wheeled inverted pendulum systems in structured environments.

While this prior research established a solid foundation for the stabilization of self-balancing robotic platforms, it did not explicitly address the challenges introduced by variable payloads, dynamically changing system parameters, or unexpected interactions with humans. These factors are critical in practical human–robot interaction scenarios, where payload properties may be unknown or time-varying and external disturbances can significantly affect system behavior.

The present thesis extends this earlier work by explicitly considering variable payload conditions and dynamic uncertainties within the control design. By building upon the insights and experimental framework developed by the former senior student, this research advances the state of the art toward more robust, adaptive, and safety-oriented control strategies suitable for real-world human-centric environments.
         