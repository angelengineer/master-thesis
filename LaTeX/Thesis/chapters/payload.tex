\section{Payload Mass and Center-of-Mass Estimation}

This section presents a quasi-static estimation method for the payload mass and its horizontal distance from the arm joint axis. The payload is modeled as a point mass, and the estimation relies on reaction force and torque measurements obtained from the lift linear actuator and the arm motor.

\subsection{Assumptions}

The following assumptions are made:
\begin{itemize}
    \item The payload is rigidly attached to the fork and can be approximated as a point mass.
    \item The system operates under quasi-static conditions.
    \item Reaction estimates are available from the lift actuator and the arm motor.
\end{itemize}

\subsection{Available Measurements}

The estimator uses the following measured or estimated quantities:
\begin{itemize}
    \item $\hat{f}_{m}^{\mathrm{reac}}$: estimated reaction force at the lift actuator
    \item $\hat{\tau}_{a}^{\mathrm{reac}}$: estimated reaction torque at the arm motor
    \item $\theta_p^{\mathrm{res}}$: measured pitch angle
    \item $d_m^{\mathrm{res}}$: measured lift position
    \item $\theta_a^{\mathrm{res}}$: measured arm motor angle
\end{itemize}

The parameters to be estimated are:
\begin{itemize}
    \item $\hat{m}$: payload mass
    \item $\hat{d}$: horizontal distance of the payload center of mass from the arm joint axis
\end{itemize}

\subsection{Estimation Equations}

When the lift mechanism is active, the payload mass can be directly estimated from the lift reaction force as
\begin{equation}
\hat{m}_{\mathrm{lift}} =
\frac{\hat{f}_{m}^{\mathrm{reac}}}
{g \cos\!\left(\theta_p^{\mathrm{res}}\right)},
\end{equation}
where $g$ denotes the gravitational acceleration.

The payload mass can alternatively be estimated from the arm reaction torque according to
\begin{equation}
\hat{m}_{\mathrm{arm}} =
\frac{\hat{\tau}_{a}^{\mathrm{reac}}}
{g \cos\!\left(\theta_a^{\mathrm{res}} + \theta_p^{\mathrm{res}}\right)\, \hat{d}}.
\end{equation}

The distance of the payload center of mass from the arm joint axis is estimated as
\begin{equation}
\hat{d} =
\begin{cases}
\dfrac{\hat{\tau}_{a}^{\mathrm{reac}}}
{g \cos\!\left(\theta_a^{\mathrm{res}} + \theta_p^{\mathrm{res}}\right)\, \hat{m}},
& \text{if } d_m^{\mathrm{res}} > 0, \\[10pt]
d_{\mathrm{fork}}^{\mathrm{nom}},
& \text{otherwise},
\end{cases}
\end{equation}
where $d_{\mathrm{fork}}^{\mathrm{nom}}$ denotes the nominal center-of-mass distance of the empty fork.

\subsection{Estimation Logic}

The final payload mass estimate is selected based on the lift position:
\begin{equation}
\hat{m} =
\begin{cases}
\hat{m}_{\mathrm{lift}}, & \text{if } d_m^{\mathrm{res}} > 0, \\
\hat{m}_{\mathrm{arm}},  & \text{otherwise}.
\end{cases}
\end{equation}

\subsection{Implementation Remark}

The estimation logic is implemented using Simulink blocks and logical switching. By combining force-based and torque-based measurements, the estimator remains operational even when one of the reaction signals is unavailable, ensuring robustness across different operating conditions.
