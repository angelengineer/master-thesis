
\noindent\textbf{Abstract.} This thesis addresses the stabilization and control of underactuated self-balancing wheeled robots operating with variable payloads in physical human–robot collaboration scenarios. The research focuses on the Two-Wheeled Forklift Robot (TWFR), a differential-drive platform with a vertically actuated fork mechanism that introduces time-varying system parameters during payload manipulation. A comprehensive control framework is developed integrating Nonlinear Model Predictive Control (NMPC) with disturbance observation techniques. The NMPC, implemented through \emph{acados} using Real-Time Iteration schemes, simultaneously regulates longitudinal velocity and pitch stabilization while enforcing operational constraints. Two disturbance observers—a Disturbance Observer (DOB) for external forces and a Reaction Disturbance Observer (RDOB) for payload-induced torques—provide real-time estimates of payload mass and center of gravity without physical sensors, continuously updating the NMPC internal model for adaptive control during manipulation. A novel collision detection mechanism analyzes prediction error derivatives to identify unexpected external interactions despite varying payloads, while a supervisory Finite State Machine coordinates operational modes across normal operation, payload handling, and collision response scenarios. Comprehensive simulations validate the proposed framework with payloads up to 10~kg under diverse conditions, including trajectory tracking, external disturbances, and collision events. Results demonstrate stable balancing, accurate tracking, effective disturbance rejection, reliable collision detection, and real-time computational feasibility, advancing self-balancing robot control by addressing underactuation, payload variability, and interaction safety within a unified predictive framework suitable for real-world collaborative environments.

\vspace{1em}
\noindent\textbf{Keywords:} Self-balancing robot, wheeled inverted pendulum, underactuated control, nonlinear model predictive control, disturbance observer, variable payload, collision detection, human–robot interaction
