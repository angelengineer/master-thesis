This chapter presents the control architecture adopted for the TWFR, which combines a centralized nonlinear model predictive controller with disturbance observer-based techniques. The overall strategy is designed to address the strongly nonlinear and underactuated nature of the system while ensuring robust and safe operation during human--robot interaction. A single nonlinear model predictive control (NMPC) scheme is employed to simultaneously regulate the longitudinal motion of the robot along the $x$-axis and stabilize its pitch angle, which is essential for maintaining dynamic balance. By explicitly incorporating the nonlinear system dynamics and physical constraints, NMPC provides anticipative control actions and naturally acts as a prediction-based safety layer, enabling early detection of abnormal behaviors and appropriate reactions to external interactions. To further enhance robustness and situational awareness, disturbance and reaction disturbance observers (DOB/RDOB) are integrated as virtual sensing mechanisms. These observers provide real-time estimates of external disturbances and payload-induced uncertainties without relying on additional physical sensors, thereby complementing the NMPC framework and improving control performance under varying operating conditions.
