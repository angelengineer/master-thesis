\section{Finite State Machine}

Distinguishing between disturbances caused by charging or discharging operations and those arising from physical contact is non-trivial, as both scenarios produce similar sensor signatures. To address this challenge, a Finite State Machine (FSM) is proposed to robustly discriminate between these situations and to enforce appropriate system behavior.

\subsection{FSM Inputs}

The FSM operates based on the following boolean input signals:
\begin{itemize}
    \item \textbf{Start}: Enables the FSM operation.
    \item \textbf{Move}: Commands the robot to move according to the provided reference.
    \item \textbf{Load}: Indicates that a loading or unloading operation is in progress.
    \item \textbf{Collision}: Signals the detection of an external disturbance or collision.
\end{itemize}

\subsection{FSM States}

\begin{figure}[t]
    \centering
    \includegraphics[width=0.5\linewidth]{FSM.png}
    \caption{Block diagram of the finite state machine (FSM) implementation.}
    \label{fig:FSM}
\end{figure}

The FSM is composed of four discrete states, each associated with a specific robot behavior:
\begin{itemize}
    \item \textbf{$S_0$ -- Idle:} The robot remains stationary with no active motion or task execution.
    \item \textbf{$S_1$ -- Moving:} The robot moves while tracking the given motion references.
    \item \textbf{$S_2$ -- Loading:} The robot remains stationary and deliberately ignores disturbance signals to allow safe loading or unloading.
    \item \textbf{$S_3$ -- Collision:} Upon detecting a collision, the robot performs a short backward motion and subsequently stops.
\end{itemize}

\subsection{State Transitions}

The state transitions are governed by the following transition functions:
\begin{itemize}
    \item \textbf{$f_{0,1}^{\sigma}$:} $\mathrm{cmd}(\text{Move}) = 1$
    \item \textbf{$f_{0,2}^{\sigma}$:} $\mathrm{cmd}(\text{Load}) = 1$
    \item \textbf{$f_{1,2}^{\sigma}$:} $\mathrm{cmd}(\text{Load}) = 1$
    \item \textbf{$f_{1,3}^{\sigma}$:} $\mathrm{cmd}(\text{Collision}) = 1$
    \item \textbf{$f_{2,1}^{\sigma}$:} $\mathrm{cmd}(\text{Move}) = 1$
    \item \textbf{$f_{3,0}^{\sigma}$:} $\mathrm{cmd}(\text{Start}) = 1$
\end{itemize}

The resulting FSM diagram is illustrated in Figure.
% TABLE 1: FSM STATES (Professional B&W)
\begin{table}[h]
\centering
\small
\begin{tabular}{@{}llp{4.2cm}@{}}
\toprule
\textbf{State} & \textbf{Behavior} & \textbf{Key Feature} \\
\midrule
$S_0$ -- Idle & 
Stationary, no motion execution & 
System entry/exit point \\
$S_1$ -- Moving & 
Tracks motion references while monitoring disturbances & 
Active navigation mode \\
$S_2$ -- Loading & 
Stationary with \textbf{disturbance signals ignored} & 
Safe human-robot interaction during load/unload \\
$S_3$ -- Collision & 
Short backward motion $\rightarrow$ full stop & 
Automatic safety recovery \\
\bottomrule
\end{tabular}
\caption{FSM operational states}
\label{tab:fsm_states}
\end{table}

% TABLE 2: TRANSITION FUNCTIONS (Professional B&W - CORRECTED)
\begin{table}[h]
\centering
\small
\begin{tabular}{@{}>{$}l<{$}>{$}l<{$}p{4.5cm}@{}}
\toprule
\textbf{Transition} & \textbf{Condition} & \textbf{Description} \\
\midrule
f_{0,1}^{\sigma} & \mathrm{cmd}(\text{Move}) & Start navigation from idle \\
f_{0,2}^{\sigma} & \mathrm{cmd}(\text{Load}) & Initiate loading procedure \\
f_{1,2}^{\sigma} & \mathrm{cmd}(\text{Load})  & Pause motion for loading \\
f_{1,3}^{\sigma} & \mathrm{cmd}(\text{Collision}) & Emergency response during motion \\
f_{2,1}^{\sigma} & \mathrm{cmd}(\text{Move}) & Resume navigation after loading \\
f_{3,0}^{\sigma} & \mathrm{cmd}(\text{Start})  & System reset after collision \\
\bottomrule
\end{tabular}
\caption{FSM state transition logic}
\label{tab:fsm_transitions}
\end{table}