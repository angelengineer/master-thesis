\subsection{Nonlinear Model Predictive Control}

The NMPC simultaneously regulates the longitudinal velocity and stabilizes the pitch angle while explicitly enforcing physical constraints. The continuous-time optimal control problem is formulated as:
\begin{equation}
\begin{aligned}
\min_{x(\cdot), u(\cdot)} \quad
& \int_0^T \ell(x(t), u(t)) \, \mathrm{d}t + M(x(T)) \\
\text{subject to} \quad
& \dot{x}(t) = f(x(t), u(t)), \\
& x(0) = \bar{x}_0, \\
& g(x(t), u(t)) \leq 0 .
\end{aligned}
\end{equation}

The stage cost adopts a nonlinear least-squares formulation with state weighting
\begin{equation}
W_x = \mathrm{diag}(10^{-9}, 10^{1}, 10^{2}, 10^{3}),
\end{equation}
which strongly prioritizes pitch stabilization. The control input is weighted using $W_u = 10^{-1} I_2$.

The NMPC is implemented using \textit{acados} and solved via Real-Time Iteration (RTI) Sequential Quadratic Programming. The prediction horizon consists of $N=40$ discretization steps over $T=1$~s, corresponding to a sampling time of $25$~ms. Box constraints limit the wheel torques to $\pm 10$~Nm and the pitch angle to $\pm 45^\circ$.

\subsection{Disturbance Observers}

\subsubsection{Fork Disturbance Observer (FDOB)}

The FDOB estimates lumped disturbances acting on the fork joint, including dynamic coupling effects, parametric uncertainties, and external torques. Assuming a nominal inertia parameter $m_{n44}$, the fork dynamics are expressed as:
\begin{equation}
m_{n44}\ddot{\theta}_a = n_a\tau_a - \tilde{\tau}_{a}^{\mathrm{dist}} .
\end{equation}

The disturbance is estimated using pseudo-differentiation:
\begin{equation}
\hat{\tau}_{a}^{\mathrm{dist}}
=
\frac{g_a}{s + g_a}
\left(
n_a\tau_a
+
g_a m_{n44}\dot{\theta}_a
\right)
-
g_a m_{n44}\dot{\theta}_a ,
\end{equation}
where $g_a$ denotes the observer bandwidth. The estimated disturbance is fed forward to augment a PD controller:
\begin{equation}
\tau_a =
K_{pa}(\theta_a^{\mathrm{cmd}} - \theta_a)
+
K_{da}(\dot{\theta}_a^{\mathrm{cmd}} - \dot{\theta}_a)
+
\hat{\tau}_{a}^{\mathrm{dist}} .
\end{equation}

\subsubsection{Fork Reaction Torque Observer (FRTOB)}

The FRTOB isolates the payload-induced reaction torque from the total disturbance estimate. By excluding gravity and friction effects, the reaction torque is obtained as:
\begin{equation}
\hat{\tau}_a^{\mathrm{reac}}
=
\frac{g_r}{s + g_r}
\left(
n_a\tau_a
+
g_r m_{n44}\dot{\theta}_a
-
g_4
-
\tilde{\tau}_a^{\mathrm{fric}}
\right)
-
g_r m_{n44}\dot{\theta}_a ,
\end{equation}
where $g_r$ is the observer bandwidth. Analogous observers (LDOB/LRFOB) are employed to estimate lift reaction forces.

\subsection{Payload Parameter Estimation}

The payload mass is estimated from the lift reaction force when the actuator is active:
\begin{equation}
\hat{m}_{\mathrm{lift}} =
\frac{\hat{f}_{m}^{\mathrm{reac}}}
{g \cos(\theta_p)} .
\end{equation}

The horizontal distance between the fork joint and the payload center of mass is estimated from the arm reaction torque:
\begin{equation}
\hat{d} =
\frac{\hat{\tau}_{a}^{\mathrm{reac}}}
{g \cos(\theta_a + \theta_p)\, \hat{m}} .
\end{equation}

These estimates are used to update the equivalent parameters $(M, x_G, z_G, I_{\mathrm{eq}})$ of the internal NMPC model, enabling continuous adaptation to changing payload conditions.

\subsection{Collision Detection}

Collision detection exploits the predictive capability of the NMPC. The prediction error is defined as:
\begin{equation}
e_k = x_k - \hat{x}_{k|k-1},
\end{equation}
where $\hat{x}_{k|k-1}$ denotes the one-step-ahead state prediction. The time derivative of the absolute prediction error,
\begin{equation}
\dot{e}_k = \frac{\mathrm{d}}{\mathrm{d}t} \left( | e_k | \right),
\end{equation}
emphasizes abrupt deviations from nominal behavior. A collision is detected when:
\begin{equation}
|\dot{e}_{\dot{\theta_p},k}| > \delta_{\mathrm{coll}} .
\end{equation}
In self-balancing systems, the pitch-rate prediction error is particularly sensitive to external contacts.

\begin{figure}[t]
    \centering
    \includegraphics[width=\columnwidth]{control-collision.png}
    \caption{Control architecture for collision handling.}
    \label{fig:control-collision}
\end{figure}

\subsection{Finite State Machine}

A supervisory finite state machine (FSM) coordinates the operational modes (\emph{Idle}, \emph{Moving}, \emph{Loading}, and \emph{Collision}) based on boolean inputs (\emph{Start}, \emph{Move}, \emph{Load}, \emph{Collision}). During the \emph{Loading} state, disturbance signals are ignored to ensure safe payload handling. Upon collision detection, the robot executes a backward motion and subsequently stops (\figref{fig:control-fsm}).

\begin{figure}[t]
    \centering
    \includegraphics[width=\columnwidth]{FSM.png}
    \caption{Finite state machine.}
    \label{fig:control-fsm}
\end{figure}

\begin{figure}[t]
    \centering
    \includegraphics[width=\columnwidth]{control-global-diagram.png}
    \includegraphics[width=\columnwidth]{control-FSM.png}
    \caption{Global control architecture of the system.}
    \label{fig:control-architecture}
\end{figure}
